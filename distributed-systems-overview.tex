\documentclass[10pt,a4paper]{article}

\usepackage{hyperref}
\usepackage[style=reading, sorting=ynt]{biblatex}
\addbibresource{bib/bibliography.bib}
\usepackage{csquotes}

\begin{document}

\title{Distributed Systems Overview}
\author{Jason Cairns}
\maketitle{}

\tableofcontents{}

\section{Hadoop}
\label{sec:hadoop-1}
\nocite{dean2004mapreduce}
\nocite{ghemawat2003google}
\nocite{shvachko2010hadoop}
\nocite{zheludkov2017high}
\nocite{akidau2013millwheel}
Apache Hadoop is a collection of utilities that facilitates cluster
computing. Jobs can be sent for parallel processing on the cluster
directly to the utilities using .jar files, ``streamed'' using any
executable file, or accessed through language-specific APIs.

The project began in 2006, by Doug Cutting, a Yahoo employee, and Mike
Cafarella. The inspiration for the project was a paper from Google
describing the Google File System, which was followed by another
Google paper detailing the MapReduce programming model.

Hadoop consists of a memory part, known as Hadoop Distributed File
System (HDFS), described in \ref{sec:hdfs}, and a processing part,
known as MapReduce, described in \ref{sec:mapreduce}.

In operation, Hadoop splits files into blocks, then distributes them
across nodes in a cluster, where they are then processed by the node.
This creates the advantage of data locality, wherein data is processed
by the node they exist in.

Hadoop has seen extensive industrial use as the premier big data
platform upon it's release. In recent years it has been overshadowed
by Spark, due to the greater speed gains offered by spark. The key
reason Spark is so much faster than Hadoop comes down to their
different processing approaches: Hadoop MapReduce requires reading
from disk and writing to it, for the purposes of fault-tolerance,
while Spark can run processing entirely in-memory. However, in-memory
MapReduce is provided by another Apache project, Ignite.

\subsection{Hadoop Distributed File System}
\label{sec:hdfs}

The file system has 5 primary services.

\begin{description}
\item[Name Node] Contains all of the data and manages the system. The
  master node.
  \item[Secondary Name Node] Creates checkpoints of the metadata from
  the main name node, to potentially restart the single point of
  failure that is the name node. Not the same as a backup, as it only
  stores metadata.
\item[Data Node] Contains the blocks of data. Sends ``Heartbeat
  Message'' to the name node every 3 seconds. If two minutes passes
  with no heartbeat message from a particular data node, the name node
  marks it as dead, and sends it's blocks to another data node.
\item[Job Tracker] Receives requests for MapReduce from the client,
  then queries the name node for locations of the data.
\item[Task Tracker] Takes tasks, code, and locations from the job
  tracker, then applies such code at the location. The slave node for
  the job tracker.
\end{description}

HDFS is written in Java and C.

\subsection{MapReduce}
\label{sec:mapreduce}

MapReduce is a programming model consisting of map and reduce staps,
alongside making use of keys.

\begin{description}
\item[Map] applies a ``map'' function to a dataset, in the
  mathematical sense of the word. The output data is temporarily
  stored before being shuffled based on output key, and sent to the
  reduce step.
\item[Reduce] produces a summary of the dataset yielded by the map operation
\item[Keys] are associated with the data at both steps. Prior to the
  application of mapping, the data is sorted and distributed among
  data nodes by the data's associated keys, with each key being mapped
  as a logical unit. Likewise, mapping produces output keys for the
  mapped data, and the data is shuffled based upon these keys, before
  being reduced.
\end{description}

After sorting, mapping, shuffling, and reducing, the output is
collected, sorting by the second keys and given as final output.

The implementation of MapReduce is provided by the HDFS services of
job tracker and task tracker. The actual processing is performed by
the task trackers, with scheduling using the job tracker, but other
scheduling systems are available to be made use of.

Development at Google no longer makes as much use of MapReduce as they
originally did, using stream processing technologies such as
MillWheel, rather than the standard batch processing enabled by
MapReduce.

\section{Spark}
\label{sec:spark}
\nocite{zaharia2010spark}
\nocite{zaharia2016apache}
Spark is a framework for cluster computing. Much of it's definition is
in relation to Hadoop, which it intended to improve upon in terms of
speed and usability for certain tasks.

It's fundamental operating concept is the Resiliant Distributed
Dataset (RDD), which is immutable, and generated through external
data, as well as actions and transformations on prior RDD's. The RDD
interface is exposed through an API in various languages, including R.

Spark requires a distributed storage system, as well as a cluster
manager; both can be provided by Hadoop, among others.

Spark is known for possessing a fairly user-friendly API, intended to
improve upon the MapReduce interface. Another major selling point for
Spark is the libraries available that have pre-made functions for
RDD's, including many iterative algorithms. The availability of
broadcast variables and accumulators allow for custom iterative
programming.

Spark has seen major use since it's introduction, with effectively all
major big data companies having some use of Spark.

\section{H2O}
\label{sec:h2o}
\nocite{h2o.ai:_h2o}
\nocite{h2o.ai:_home_open_sourc_leader_ai}
The H2O software bills itself as,

\begin{displayquote}
  an in-memory platform for distributed, scalable machine learning.
  H2O uses familiar interfaces like R, Python, Scala, Java, JSON and
  the Flow notebook/web interface, and works seamlessly with big data
  technologies like Hadoop and Spark. H2O provides implementations of
  many popular algorithms such as GBM, Random Forest, Deep Neural
  Networks, Word2Vec and Stacked Ensembles. H2O is extensible so that
  developers can add data transformations and custom algorithms of
  their choice and access them through all of those clients.
\end{displayquote}

H2O typically runs on HDFS, along with spark for computation and
bespoke data structures serving as the backbone of the architecture.

H2O can communicate with R through a REST api. Users write functions
in R, passing user-made functions to be applied on the objects
existing in the H2O system.

The company H2O is backed by \$146M in funding, partnering with large
institutions in the financial and tech world. Their business model
follows an open source offering with the same moniker as the company,
and a small set of heavily-marketed proprietary software in aid of it.
They have some important figures working with them, such as Matt
Dowle, creator of data.table.

\section{R Packages}
\label{sec:r-packages}

\subsection{DistributedR}
\label{sec:distributedr}
DistributedR offers cluster access for various R data structures,
particularly arrays, and providing S3 methods for a fair range of
standard functions. It has no regular cluster access interface, such
as with Hadoop or MPI, being made largely from scratch.

The package creators have ceased development as of December 2015. The
company, Vertica, has moved on to offering an enterprise database
platform.

\subsection{Foreach and DoX}
\label{sec:foreach-dox}
\nocite{microsoft20}
\nocite{corporation19}
\nocite{dosnow19}
\nocite{weston17}
Foreach offers a high-level looping construct compatible with a
variety of backends. The backends are provided by other packages,
typically named with some form of ``Do\textit{X}''. Parallelisation is
enabled by some backends, with doParallel allowing parallel
computations, and doMPI allowing for direct MPI access.

Foreach is managed by Revolution Analytics, with many of the
Do\textit{X} corollary packages also being produced by them.

\subsection{Future and Furrr}
\label{sec:future-furrr}
\nocite{bengtsson20}
\nocite{vaughan18}
Future captures R expressions for evaluation, allowing them to be
passed on for parallel and ad-hoc cluster evaluation, through the
parallel package. Such parallelisation uses the standard MPI or SOCK
protocols.

The author of future is Henrik Bengtsson, Associate Professor at UCSF.
Development on the package remains strong, with Dr.~Bengtsson
possessing a completely full commit calendar and 81,328 contributions
on GitHub.

Furrr is a frontend to future, amending the functions from the package
purrr to be compatible with future, thus enabling parallelisation in a
similar form to multicore, though with a tidyverse style.

Furrr is developed by Davis Vaughn, an employee at RStudio.

\subsection{Parallel, Snow, and Multicore}
\label{sec:parall-snow-mult}
Parallel is a package included with R, born from the merge of the
packages snow and multicore. Parallel enables various means of
performing computations in R in parallel, allowing not only multiple
cores in a node, but multiple nodes through snow's interfaces to MPI
and SOCK.

Parallel takes from multicore the ability to perform multicore
processing, with the mcapply function. Multicore creates forked R
sessions, which is very resource-efficient, but not supported by
windows.

From snow, distributed computing is enabled for multiple nodes.

Multicore was developed by Simon Urbanek (!). Snow was developed by
Luke Tierney, a professor at the University of Iowa, who also
originated the byte-compiler for R

\subsection{Partools}
\label{sec:partools}
\nocite{matloff16:_softw_alchem}
Partools provide utilities for the parallel package. It offers
functions to split files and process the splits across nodes provided
by parallel, along with bespoke statistical functions.

It consists mainly of wrapper functions, designed to follow it's
philosophy of ``keep it distributed''.

It is authored by Norm Matloff, a professor at UC, Davis.

\subsection{PBDR}
\label{sec:pbdr}
\nocite{pbdBASEpackage}
\nocite{pbdBASEvignette}
pbdR is a collection of packages allowing for distributed computing
with R. The packages include communication and computation
capabilities, including RPC, ZeroMQ, and MPI interfaces.

\subsection{RHadoop}
\label{sec:rhadoop}
\nocite{analytics:_rhadoop_wiki}
RHadoop is a collection of five packages to run Hadoop directly from
R. The packages are divided by logical function, including rmr2, which
runs MapReduce jobs, and rhdfs, which can access the HDFS. The
packages also include plyrmr, which makes available plyr-like data
manipulation functions, in a similar vein to sparklyr.

It is offered and developed by Revolution Analytics.
  
\subsection{RHIPE and DeltaRho}
\label{sec:rhipe-deltarho}
\nocite{deltarho:_rhipe}
RHIPE is a means of ``using Hadoop from R''. The provided functions
primarily attain this through interfacing and manipulating HDFS, with
a function, rhwatch, to submit MapReduce jobs. The easiest means of
setup for it is to use a VM, and for all Hadoop computation, MapReduce
is directly programmed for by the user.

There is currently no support for the most recent version of Hadoop,
and it doesn't appear to be under active open development, with the
last commit being 2015. RHIPE has mostly been subsumed into the
backend of DeltaRho, a simple frontend.
  
\subsection{Rmpi}
\label{sec:rmpi}
\nocite{yu02:_rmpi}
Rmpi is an R interface to MPI. As such, the capabilities enabled by
the package are best suited for supercomputers, such as Beowulf
clusters or Cray.

It is authored by Hao Yu, of the University of Western Ontario.

\subsection{Sparklyr}
\label{sec:sparklyr}
\nocite{luraschi20}
Sparklyr is an interface to Spark from within R. The user connects to
spark and accumulates instructions for the manipulation of a Spark
DataFrame object using dplyr commands, then executing the request on
the Spark cluster.

Sparklyr is managed and maintained by RStudio, who also manage the
rest of the Tidyverse (including dplyr).

\subsection{SparkR}
\label{sec:sparkr}
\nocite{venkataraman20:_spark}
SparkR provides a front-end to Spark from R. Like Sparklyr, it
provides the DataFrame as the primary object of interest. However,
there is no support for the dplyr model of programming, with functions
closer resembling base R being provided by the package instead.

SparkR is maintained directly by Apache Spark, with ongoing regular
maintenance provided.

\section{R Derivatives}
\label{sec:r-derivatives}

\subsection{IBM Big R}
\label{sec:ibm-big-r}
\nocite{inc.14:_infos_bigin_big_r}
IBM Big R is a library of functions integrating R with the IBM
InfoSphere BigInsights platform. This is implemented through the
introduction of various bigr classes replicating base R types. These
are then run in the background on the BigInsights platform, which is
in turn powered by Apache Hadoop. The user is therefore able to run
MapReduce jobs without having to explicitly write MapReduce-specific
code.

\subsection{MapR}
\label{sec:mapr}
\nocite{mapr19:_indus_next_gener_data_platf_ai_analy}
MapR initially provided R access to Hadoop, being mainly HDFS access.
It was then bought out by HP in May 2019, pivoting to selling an
enterprise database platform and analytics services, running on Hadoop
among other backends. Development has ceased on R access.

\defbibfilter{other}{keyword=ignite or keyword=millwheel}
\printbibheading{}
\printbibliography[keyword=hadoop, title={Hadoop}]
\printbibliography[keyword=spark,title={Spark}]
\printbibliography[keyword=h2o,title=H2O]
\printbibliography[keyword=rpackage,title={R Packages}]
\printbibliography[keyword=rderiv,title={R Derivatives}]
\printbibliography[filter=other,title=Other]
\end{document}
