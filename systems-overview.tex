\documentclass[10pt,a4paper]{article}
\usepackage[style=apa]{biblatex}
\addbibresource{bib/bibliography.bib}
\begin{document}
\title{Systems Overview}
\maketitle{}

\section{Software}
\label{sec:software}

\subsection{Hadoop}
\label{sec:hadoop-1}

Apache Hadoop is a collection of utilities that facilitates cluster
computing. Hadoop consists of a memory part, known as Hadoop
Distributed File System (HDFS), and a processing part, known as MapReduce.

In operation, Hadoop splits files into blocks, then distributes them
across nodes in a cluster, where they are then processed by the node.
This creates the advantage of data locality, wherein data is processed
by the node they exist in.

\subsubsection{Hadoop Distributed File System}
\label{sec:hdfs}

The file system has 5 primary services.

\begin{description}
\item[Name Node] Contains all of the data and manages the system. The
  master node.
  \item[Secondary Name Node] Creates checkpoints of the metadata from
  the main name node, to potentially restart the single point of
  failure that is the name node. Not the same as a backup, as it only
  stores metadata.
\item[Data Node] Contains the blocks of data. Sends ``Heartbeat
  Message'' to the name node every 3 seconds. If two minutes passes
  with no heartbeat message from a particular data node, the name node
  marks it as dead, and sends it's blocks to another data node.
\item[Job Tracker] Receives requests for MapReduce from the client,
  then queries the name node for locations of the data.
\item[Task Tracker] Takes tasks, code, and locations from the job
  tracker, then applies such code at the location. The slave node for
  the job tracker.
\end{description}

HDFS is written in Java and C.

\subsubsection{MapReduce}
\label{sec:mapreduce}

MapReduce is a programming model consisting of map and reduce staps,
alongside making use of keys.

\begin{description}
\item[Map] applies a ``map'' function to a dataset, in the
  mathematical sense of the word. The output data is temporarily
  stored before being shuffled based on output key, and sent to the
  reduce step.
\item[Reduce] produces a summary of the dataset yielded by the map operation
\item[Keys] are associated with the data at both steps. Prior to the
  application of mapping, the data is sorted and distributed among
  data nodes by the data's associated keys, with each key being mapped
  as a logical unit. Likewise, mapping produces output keys for the
  mapped data, and the data is shuffled based upon these keys, before
  being reduced.
\end{description}

After sorting, mapping, shuffling, and reducing, the output is
collected, sorting by the second keys and given as final output.

The implementation of MapReduce is provided by the HDFS services of
job tracker and task tracker. The actual processing is performed by
the task trackers, with scheduling using the job tracker, but other
scheduling systems are available to be made use of.

\subsection{Spark}
\label{sec:spark}

Spark is a framework for cluster computing. Much of it's definition is
in relation to Hadoop, which it hoped to improve upon for certain
tasks. It's fundamental concept is the Resiliant Distributed Dataset
(RDD), which is immutable, and generated through external data, as
well as actions and transformations on prior RDD's. The RDD interface
is exposed through an API in various languages, including R.

Spark requires a distributed storage system, as well as a cluster
manager; both can be provided by Hadoop, among others.

A major selling point for Spark is the libraries available that have
pre-made functions for RDD's, including many iterative algorithms.
Broadcast variables and accumulators allow for custom iterative
programming.

\subsection{H2O}
\label{sec:h2o}



\subsection{R Packages}
\label{sec:r-packages}

\begin{description}
\item[parallel] 
\item[snow] 
\end{description}

\section{Comparison between R and Other Software}
\label{sec:comparison-between-r}

\section{Questions and Notes}
\label{sec:questions}

When using Hadoop, what happens if the data associated with a
particular key is too large to fit on any one node?

Google no longer uses much of MapReduce, using stream processing
technologies such as MillWheel, rather than the batch processing of
MapReduce

\nocite{zaharia2010spark}
\nocite{shvachko2010hadoop}
\nocite{dean2004mapreduce}
\nocite{ghemawat2003google}
\printbibliography{}
\end{document}
