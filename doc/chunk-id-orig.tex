\documentclass[a4paper,10pt]{article}

\usepackage{doc/header}

\begin{document}
\title{Chunk ID Origination and Client-Server Communication}
\author{Jason Cairns}
\year=2020 \month=9 \day=3
\maketitle{}

\section{Introduction}

The problem of chunk ID origination as discussed in
\href{init-chunk-msg-q-exp.pdf}{Initial Chunk Message Queue Experiments}
dictates much of the client-server communication, as well as the state of
knowledge in the network, given that chunk ID is used as the key to send
messages relevant to a particular chunk.
This document serves to model the differences between naive client-originated
chunk ID's, and server-originated chunk ID's, with an evaluation and proposal
that aims at overcoming the limitations involved in the models.

\section{Modelling}\label{sec:cid-model}

The models consist of communication over time between a client and a server,
intermediated by a queue server.
The client runs the pseudo-program described in listing \ref{src:client-p},
where variables \texttt{x}, \texttt{y}, and \texttt{z} are chunk references,
and variables \texttt{i} and \texttt{j} are local. 
Every action on distributed chunk references entails a message to the queue
named by the associated chunk ID, requesting the relevant action to be
performed.

\begin{lstlisting}[language=R, float, 
caption={Modelled Client Program}, label=src:client-p]
	y <- f(x)	# dist, no wait
	f(i)		# local
	z <- f(y)	# dist, no wait
	f(j)		# local
	f(z)		# dist, wait
\end{lstlisting}

The server follows a loop of waiting on queues relevant to the chunks that it
stores and performing requests from the messages, through taking the function
relayed in the message and performing it on the local object identified by the
chunk ID given by the queue name the message was popped from.
Without loss of generality, the function \texttt{f} is considered to take
constant time on local objects, and messages likewise have constant latency;
the ratio of latency to operation time is irrelevent to what is demonstrated in
these models.
Assignment, message listening, and message switching by the queue are
considered to be instantaneous events.

\subsection{Client-Originated Chunk ID}

In the client-originated chunk ID  model, in addition to the generic model
description posed in section \ref{sec:cid-model}, the client sends a chunk ID
as part of its messages if the result of the function on the distributed object
includes assignment.
If there is no assignment, the message includes a job ID instead, naming a job
queue to be monitored by the client.
If the server receives a job ID, it sends the value of the computed function to
the queue with that job ID as it's key, sending no messages otherwise.

A Space-Time Diagram\cite{lamport1978ordering} of the client-originated chunk
ID model is given in figure \ref{fig:mci}.

\begin{figure}
	\begin{minipage}{0.6\textwidth}
	\begin{tikzpicture}
		\draw[very thin, -{>[scale=2]}] (0,10) -- (0,0) 
			node[at start, anchor=south] {C};
		\draw[very thin, -{>[scale=2]}] (2,10) -- (2,0) 
			node[at start, anchor=south] {Q};
		\draw[very thin, -{>[scale=2]}] (4,10) -- (4,0) 
			node[at start, anchor=south] {S};

		\draw[-{>[scale=2]}, dark2-1] (0, 9.5) -- (2,9) 
			node[at start, anchor=east] {\texttt{y <- f(x)}}
			node[midway, anchor=south] {a};
		\draw[-{>[scale=2]}, dark2-1] (2, 9) -- (4,8.5)
			node[midway, anchor=south] {b};
		\draw[very thick, dark2-1] (4,8.5) -- (4,7.5) 
			node[midway, anchor=west] {\texttt{f(C1)}};
		\draw[-{>[scale=2]}, dark2-1] (4, 7.5) -- (2,7) 
			node[midway, anchor=south] {c};
		\draw[-{>[scale=2]}, dark2-1] (2, 7) -- (0, 6.5)
			node[midway, anchor=south] {d};

		\draw[very thick, dark2-2] (0, 9.5) -- (0, 8.5)
			node[midway, anchor=east] {\texttt{f(i)}};

		\node[anchor=east, color=dark2-3] at (0,8.5) {\texttt{z <- f(y)}};
		\draw[-{>[scale=2]}, dark2-3] (0, 6.5) -- (2,6) 
			node[midway, anchor=south] {e};
		\draw[-{>[scale=2]}, dark2-3] (2,6) -- (4,5.5)
			node[midway, anchor=south] {f};
		\draw[very thick, dark2-3] (4,5.5) -- (4,4.5) 
			node[midway, anchor=west] {\texttt{f(C2)}};
		\draw[-{>[scale=2]}, dark2-3] (4,4.5) -- (2,4) 
			node[midway, anchor=south] {g};
		\draw[-{>[scale=2]}, dark2-3] (2,4) -- (0,3.5)
			node[midway, anchor=south] {h};

		\draw[very thick, dark2-4] (0, 6.5) -- (0,5.5)
			node[midway, anchor=east] {\texttt{f(j)}};
			
		\node[anchor=east, color=dark2-5] at (0,5.5) {\texttt{f(z)}};
		\draw[-{>[scale=2]}, dark2-5] (0,3.5) -- (2, 3) 
			node[midway, anchor=south] {i};
		\draw[-{>[scale=2]}, dark2-5] (2,3) -- (4,2.5)
			node[midway, anchor=south] {j};
		\draw[very thick, dark2-5] (4,2.5) -- (4,1.5) 
			node[midway, anchor=west] {\texttt{f(C3)}};
		\draw[-{>[scale=2]}, dark2-5] (4,1.5) -- (2,1) 
			node[midway, anchor=south] {k};
		\draw[-{>[scale=2]}, dark2-5] (2, 1) -- (0,0.5)
			node[midway, anchor=south] {l};
	\end{tikzpicture}
	\end{minipage}
	\begin{minipage}{0.4\textwidth}
		\begin{description}
			\item [\textcolor{dark2-1}{a}]
			\item [\textcolor{dark2-1}{b}] 
			\item [\textcolor{dark2-1}{c}] 
			\item [\textcolor{dark2-1}{d}] 
			\item [\textcolor{dark2-3}{e}] 
			\item [\textcolor{dark2-3}{f}] 
			\item [\textcolor{dark2-3}{g}] 
			\item [\textcolor{dark2-3}{h}] 
			\item [\textcolor{dark2-5}{i}] 
			\item [\textcolor{dark2-5}{j}] 
			\item [\textcolor{dark2-5}{k}] 
			\item [\textcolor{dark2-5}{l}] 
		\end{description}
	\end{minipage}
	\caption{\label{fig:mci} Space-Time Diagram of Client-Originated Chunk ID}
\end{figure}


\subsection{Server-Originated Chunk ID}

In the server-originated chunk ID model, given that the client doesn't know the
chunk ID of created chunk references, leaving that to the server, it sends out
messages with job IDs, creating chunks references that at first reference the
job ID, but when the actual chunk ID is required, waiting on the job ID queue
for a message containing it's chunk ID.
The server correspondingly sends chunk IDs of each newly assigned chunk to the
job ID queue specified in the request, sending values instead if not directed
to perform assignment.

A diagram of the server-originated chunk ID model is given in figure \ref{fig:msi}.

\begin{figure}
	\begin{minipage}{0.6\textwidth}
	\begin{tikzpicture}
		\draw[very thin, -{>[scale=2]}] (0,10) -- (0,0) 
			node[at start, anchor=south] {C};
		\draw[very thin, -{>[scale=2]}] (2,10) -- (2,0) 
			node[at start, anchor=south] {Q};
		\draw[very thin, -{>[scale=2]}] (4,10) -- (4,0) 
			node[at start, anchor=south] {S};

		\draw[-{>[scale=2]}, dark2-1] (0, 9.5) -- (2,9) 
			node[at start, anchor=east] {\texttt{y <- f(x)}}
			node[midway, anchor=south] {a};
		\draw[-{>[scale=2]}, dark2-1] (2, 9) -- (4,8.5)
			node[midway, anchor=south] {b};
		\draw[very thick, dark2-1] (4,8.5) -- (4,7.5) 
			node[midway, anchor=west] {\texttt{f(C1)}};

		\draw[very thick, dark2-2] (0, 9.5) -- (0,8.5) 
			node[midway, anchor=east] {\texttt{f(i)}};

		\draw[-{>[scale=2]}, dark2-3] (0, 8.5) -- (2,8) 
			node[at start, anchor=east] {\texttt{z <- f(y)}}
			node[midway, anchor=south] {c};
		\draw[-{>[scale=2]}, dark2-3] (2,7.5) -- (4,7)
			node[midway, anchor=south] {e};
		\draw[very thick, dark2-3] (4,7) -- (4,7) 
			node[midway, anchor=west] {\texttt{f(C2)}};

		\draw[very thick, dark2-4] (0, 8.5) -- (0,7.5)
			node[midway, anchor=east] {\texttt{f(j)}};
			
		\draw[-{>[scale=2]}, dark2-5] (0, 7.5) -- (2,7) 
			node[at start, anchor=east] {\texttt{z <- f(y)}}
			node[midway, anchor=south] {d};
		\draw[-{>[scale=2]}, dark2-5] (2,6) -- (4,5.5)
			node[midway, anchor=south] {f};
		\draw[very thick, dark2-5] (4,5.5) -- (4,4.5) 
			node[midway, anchor=west] {\texttt{f(C2)}};
		\draw[-{>[scale=2]}, dark2-5] (4,4.5) -- (2,4)
			node[midway, anchor=south] {g};
		\draw[-{>[scale=2]}, dark2-5] (2,4) -- (0,3.5)
			node[midway, anchor=south] {h};
	\end{tikzpicture}
	\end{minipage}
	\begin{minipage}{0.4\textwidth}
		\begin{description}
			\item [\textcolor{dark2-1}{a}]
			\item [\textcolor{dark2-1}{b}] 
			\item [\textcolor{dark2-3}{c}] 
			\item [\textcolor{dark2-5}{d}] 
			\item [\textcolor{dark2-3}{e}] 
			\item [\textcolor{dark2-5}{f}] 
			\item [\textcolor{dark2-5}{g}] 
			\item [\textcolor{dark2-5}{h}] 
		\end{description}
	\end{minipage}
	\caption{\label{fig:msi} Space-Time Diagram of Server-Originated Chunk ID}
\end{figure}


\section{Evaluation}\label{sec:mod-eval}

Clearly, the server-originated chunks result in significantly more waiting on
the client end, as the chunk ID needs to be attained for every operation on the
associated chunk, which is only able to be acquired after completing the
function.

The server could in theory send the chunk ID prior to performing the requested
operation, but that leads to significant issues when the operation results in
error, as it is difficult to communicate such a result back to the client after
performing the function.
Despite the reduced time spent blocking, the client-originated chunk ID
modelled also has issue with errors; consider if the \texttt{x <- f(y)} had
been faulty, with the resultant operation of \texttt{f(C1)} rendering an error.
This would not be determined by the client untile the completion of
\texttt{f(z)}, in which an error would presumably result.
Worse, if the chunk reference \texttt{x} was given as an additional argument to
another server, which in turn requested the chunk \texttt{C1} from the node
\texttt{C1} resided upon, the error would propagate, with the source of the
error being exceedingly difficult to trace.

\section{Proposal}

A potential solution to the problems of the models posed in section
\ref{sec:mod-eval} is to treat chunk reference objects somewhat like futures,
which have a state of \texttt{resolved} or \texttt{unresolved}, with failures
also encapsulated in the object upon resolution\cite{bengtsson19:_futur_r}.

If chunk ID is client-originated, then its outgoing messages can also supply a
job ID for the server to push a message to upon completion that the client can
in turn refer to, in order to check resolution status as well as any potential
errors.

This would capture the benefits of the modelled client-originated chunk ID in
reduces wait time, with the robustness of server-originated ID in signalling of
errors.
The introduction of future terminology of \texttt{resolved()}, as well as
additional slots in a chunk to determine resolution state, as well as the use
of job ID queues for more than just value returns will be sufficient to
implement such a design.
The asynchrony may lead to non-deterministic outcomes in the case of failure,
but the use of \texttt{resolved()} and it's associated barrier procedure,
\texttt{resolve()} will enable the forcing of a greater degree of synchrony,
and allow tracing of errors back to their source in a more reliable manner.

A diagram of the proposed model, along with an additional \texttt{resolved(x)}
step, is given in \ref{fig:masi}

\begin{figure}
	\begin{minipage}{0.6\textwidth}
	\begin{tikzpicture}
		\draw[very thin, -{>[scale=2]}] (0,10) -- (0,0) 
			node[at start, anchor=south] {C};
		\draw[very thin, -{>[scale=2]}] (2,10) -- (2,0) 
			node[at start, anchor=south] {Q};
		\draw[very thin, -{>[scale=2]}] (4,10) -- (4,0) 
			node[at start, anchor=south] {S};

		\draw[-{>[scale=2]}, color=dark2-1] (0, 9.5) -- (2,9) 
			node[at start, anchor=east] {\texttt{y <- f(x)}}
			node[midway, anchor=south] {a};
		\draw[-{>[scale=2]}, color=dark2-1] (2, 9) -- (4,8.5)
			node[midway, anchor=south] {b};
		\draw[very thick, color=dark2-1] (4,8.5) -- (4,7.5) 
			node[midway, anchor=west] {\texttt{f(C1)}};
		\draw[-{>[scale=2]}, color=dark2-1] (4,7.5) -- (2, 7)
			node[midway, anchor=south] {e};

		\draw[very thick, color=dark2-2] (0, 9.5) -- (0,8.5) 
			node[midway, anchor=east] {\texttt{f(i)}};

		\draw[-{>[scale=2]}, color=dark2-3] (0, 8.5) -- (2,8) 
			node[at start, anchor=east] {\texttt{z <- f(y)}}
			node[midway, anchor=south] {c};
		\draw[-{>[scale=2]}, color=dark2-3] (2,7) -- (4,6.5)
			node[midway, anchor=south] {f};
		\draw[very thick, color=dark2-3] (4,6.5) -- (4,5.5) 
			node[midway, anchor=west] {\texttt{f(C2)}};
		\draw[-{>[scale=2]}, color=dark2-3] (4,5.5) -- (2, 5)
			node[midway, anchor=south] {g};

		\draw[very thick, color=dark2-4] (0, 8.5) -- (0,7.5)
			node[midway, anchor=east] {\texttt{f(j)}};
			
		\draw[-{>[scale=2]}, color=dark2-5] (0, 7.5) -- (2,7) 
			node[at start, anchor=east] {\texttt{z <- f(y)}}
			node[midway, anchor=south] {d};
		\draw[-{>[scale=2]}, color=dark2-5] (2,5) -- (4,4.5)
			node[midway, anchor=south] {h};
		\draw[very thick, color=dark2-5] (4,4.5) -- (4,3.5) 
			node[midway, anchor=west] {\texttt{f(C2)}};
		\draw[-{>[scale=2]}, color=dark2-5] (4,3.5) -- (2,3)
			node[midway, anchor=south] {i};
		\draw[-{>[scale=2]}, color=dark2-5] (2,3) -- (0,2.5)
			node[midway, anchor=south] {j};

		\draw[-{>[scale=2]}, color=dark2-6] (0,2.5) -- (2,2)
			node[at start, anchor=east] {\texttt{resolved(x)}}
			node[midway, anchor=south] {k};
		\draw[-{>[scale=2]}, color=dark2-6] (2,2) -- (0,1.5)
			node[midway, anchor=south] {l};
	\end{tikzpicture}
	\end{minipage}
	\begin{minipage}{0.4\textwidth}
		\begin{description}
			\item [\textcolor{dark2-1}{a}]
			\item [\textcolor{dark2-1}{b}] 
			\item [\textcolor{dark2-3}{c}] 
			\item [\textcolor{dark2-5}{d}] 
			\item [\textcolor{dark2-1}{e}] 
			\item [\textcolor{dark2-3}{f}] 
			\item [\textcolor{dark2-3}{g}] 
			\item [\textcolor{dark2-5}{h}] 
			\item [\textcolor{dark2-5}{i}] 
			\item [\textcolor{dark2-5}{j}] 
			\item [\textcolor{dark2-6}{k}] 
			\item [\textcolor{dark2-6}{l}] 
		\end{description}
	\end{minipage}
	\caption{\label{fig:masi} Space-Time Diagram of Proposed Amended Server-Originated Chunk ID}
\end{figure}

\printbibliography
\end{document}

