\documentclass[a4paper,10pt]{article}

\usepackage{doc/header}

\begin{document}
\title{Further Experimentation with a Message Queue Communication System}
\author{Jason Cairns}
\year=2020 \month=8 \day=19
\maketitle{}

\section{Introduction}

Given the simplicity and promise of flexibility as demonstrated in the
documents \href{inter-node-comm-w-redis.pdf}{Inter-node Communication with
Redis} and \href{message-queues-comms.pdf}{Message Queue Communications},
further experimentation around the concept is undertaken and documented herein.
The experiments are built successively upon it's prior, with the aim of rapidly
approximating a functioning prototype via experimentation.

\section{General Function on Single Chunk}

While the RPC-based architecture as described in
\href{experiment-eager-dist-obj-pre.pdf}{Experiment: Eager Distributed Object}
had significant limitations, a particularly powerful construct was the higher
order function \texttt{distributed.do.call}, which took functions as arguments
to be performed on the distributed chunks.

This construct is powerful in that it can serve as the basis for nearly every
function on distributed chunks, and this section serves to document experiments
relating to the creation of a general function that will perform a function at
the node hosting a particular chunk.

\subsection{With Value Return}

Regardless of performing the actual function, some means of returning the value
of a function must be provided; this section focuses on getting a function to
be performed on a worker node, with the result send back to the requester.
Listings of an implementation of these concepts are given by listings
\ref{src:vr-master} and \ref{src:vr-worker}.

To this end, the requesting node has a function defined as \texttt{doFunAt(fun,
chunk)}, which takes in any function, and the name of a chunk to perform the
function on.
\texttt{doFunAt} first composes a message to send to the chunk's queue, being a
list consisting of the function, the chunk name, and a return address, which
contains sufficient information for the node performing the operation on the
chunk to send the results back to via socket connection.
The message is then serialised and pushed to the chunk's queue, and the
requesting node sits listening on the socket that it has set up and advertised.

On the chunk-containing-node end, it sits waiting on it's preassigned queues,
each of which correspond to a chunk that it holds. Upon a message coming
through, it runs a \texttt{doFun} function on the message, which in turn runs
the function on the chunk named in the message. 
It then creates a socket connected to the requesters location as advertised in
the message, and sends the serialised results through.

\lstinputlisting[language=R,float,
	caption={Value return to request for Master Node, from original source at \lstname},
	label=src:vr-master]{../R/val-return-msg-q-master.R}

\lstinputlisting[language=R,float,
	caption={Value return to request for Worker Node, from original source at \lstname},
	label=src:vr-worker]{../R/val-return-msg-q-worker.R}

A problem with this approach is the fickle aspect of creating and removing
sockets for every request; beyond the probability of missed connections and
high downtime due to requester waiting on a response, R only has a very limited
number of connections available to it, so it is impossible to scale beyond that
limit.

\subsection{With Assignment}

Assigning the results of distributed operation to a new chunk is a far more
common operation in a distributed system in order to minimise data movement.
This will involve specifying additional directions as part of the request
message, in order to specify that assignment, and not merely the operation, is
desired.

The first issue that arises in assignment is the nature of knowledge in the
system.
Specifically, the origination of the name of the new reference; does the
requesting node generate the name, passing that on as part of the request to
the worker, or does the worker generate the name, passing that back as
information to the requester?
While it may seem arbitrary, it has potentially significant effects on the
capacity of the system.
For example, were the requester to generate the name itself, it has the
potential to generate a reference to the new distributed object with correct
names and queues to post messages to, before the worker has completed creation
of the chunks, thus forming a type of eager and asynchronous object creation on
the requesting node's end.
If it were the alternative, in the master having no knowledge of the name of
the new chunk, it would have to wait for that information before proceeding
with the creation of the object.
It could proceed with creation and delegate the task of attaining the names of
chunks to a separate service, which in turn would require a considered portion
of the architecture.
This experiment will therefore involve the implementation of master-originated
names, before considering what form worker-originated names may take.

The actual creation of a chunk name in itself demands a system-wide unique
identifier; this is a solved problem with a central message server, in redis
providing an \texttt{INCR} operation, which can be used to generate a new chunk
ID that is globally unique.

It will be clear from the previous example that the problem of inter-node data
movement, somewhat solved via direct sockets in that previous example, is
largely an implementation issue, and a problem entirely distinct to the
remainder of the logic of the system.
From this experiment onwards, the mechanism of data movement is abstracted out,
with the assumption that there will exist some additional tool that can serve
as a sufficient backend for data movement.
In reality, until that tool is developed, data will be sent through redis; not
a solution, but something that can be ignored without loss of generality.

% Implementation complete, examples follow....
% issue of nulls/0-length resultant chunks?? and what if that is in fact expected??

\section{General Function on Multiple Nodes}
\section{Data Transfer between Nodes for Multivariate Functions}
\subsection{Optimisations}
Self-environment search, hashing cache potential
\section{Alignment Mechanics}
\section{Data Origination}
Hadoop, split csvs, etc.

\end{document}
