\documentclass[a4paper,10pt]{article}

\usepackage{doc/header}

\begin{document}
\title{Application Performance Management}
\author{Jason Cairns}
\year=2021 \month=2 \day=19
\maketitle{}

\section{Introduction}

Upon attaining a level of complexity greater than trivial, a performance
management system becomes essential for understanding and aiding in the runtime
control as well as development of the platform.
Most mainstream distributed computing systems come bundled with performance
management utilities, and a case study of Hadoop and Spark is considered in
section \ref{hscs}.
The performance management system required for largeScaleR must be largely
monitoring-focussed, with a simple architecture capable of adapting to changes
in the architecture of the main platform.
The user interface will necessarily differ in some respects and emphases from
existing systems, in order to accomodate the unique architecture of
largeScaleR, with considerations on the valid metrics, control, and access
explored in sections \ref{metr}, \ref{conto}, and \ref{acces} respectively.

\section{Case Study: Hadoop \& Spark}\label{hscs}

Both Hadoop and Spark include extensive monitoring and control systems,
centralised in a convenient user-friendly web interface.
The origin of their monitoring systems is through the various Hadoop daemons
and Spark executors.
The Java Metrics framework is used by both for the generation of metrics,
triggered by any event.
The metrics are stored in event logs, which are then parsed and available for
reporting.
The web interface is decoupled from the information necessary for reporting,
and it is common practice to use alternative user interfaces such as Ganglia or
DataDog.

% https://spark.apache.org/docs/latest/monitoring.html && https://hadoop.apache.org/docs/r3.2.2/hadoop-project-dist/hadoop-common/Metrics.html
% uses Metrics (https://metrics.dropwizard.io/) framework
% can use external interfaces such as ganglia (http://ganglia.sourceforge.net/) or datadog (http://ganglia.sourceforge.net/)

\section{Metrics}\label{metr}

\section{Control}\label{conto}

\section{Access: UI \& API}\label{acces}

REST API

\printbibliography
\end{document}
