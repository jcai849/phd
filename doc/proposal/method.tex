The principal methodology to be utilised in this research is to adopt a research software development approach, informed by statistical ends\cite{bezroukov1999open}.
This includes experimental research and rapid prototyping, along with open-source practices for public feedback.
Constraints of such an approach include the many software development practices being intended primarily for teams of software developers, which is not the case in this project, as well as the need to engage in marketing in order to have any broad feedback on an open-source project.
An immediate technical challenge that exists is the staggering array of potential technologies that could be used, coupled with the myriad niche demands of end users sitting at varying stages of their respective technologies' hype cycles.
This is typically overcome through experience but failing that, an emphasis on communication and high levels of background research can be used to manage such an uncertainty.
The central use of R also presents it's own challenges, but these are often surmounted through a foreign language interface, such as C\cite{wickham2019advancedr}.

Further work will continue from the existing prototype, which is described in turn in section \ref{curr} below.
Progress on written PhD work and experimentation is freely accessible from the \href{https://github.com/jcai849/phd}{jcai849/phd} GitHub repository, and work on the LargeScaleR package is available from the \href{https://github.com/jcai849/phd}{jcai849/largeScaleR} GitHub repository\cite{cairns2020largescaler}.
