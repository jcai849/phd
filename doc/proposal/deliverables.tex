All provisional goals have been completed, with the goal of proposal approval being contingent on the committee.
A description of the goals and statements on their completion follows:

\begin{enumerate}
        \item Approval of the full thesis proposal by the appropriate departmental/faculty postgraduate committee.
                This will be granted contingent on the reception of this proposal.
        \item A substantial piece of written work, such as a literature review, completed to the satisfaction of the main supervisor.
                A 15,000 word literature review document has been produced, at a draft stage that is completed to the satisfaction of the main supervisor.
                This can be found in the \texttt{phd} git repository under \texttt{doc/lit-review.tex}.
        \item Ethics approval/s and/or permissions obtained for the research (if required).
                No ethics approval or permissions are required
        \item Attendance at one of the Doctoral Skills Programme Induction Days.
                This was completed on 2020-05-29.
        \item Successful completion of the Academic Integrity Module.
                This was completed during undergraduate study.
        \item A needs analysis to determine training and other requirements that must be completed before candidature can be confirmed.
                This was completed, with the document presented to the PYR committee.
        \item Completion of a health and safety risk assessment and training for any laboratory/studio/field and related work activities.
                No health and safety training was required for the type of work needed for this project.
        \item Undertake Diagnostic English Language Needs Assessment (DELNA) online screening.
                If a full assessment is advised, complete full diagnostic test and participate in any language enrichment recommended by the DELNA Language Advisor.
                This was completed during undergraduate study.
        \item Write a review of existing methods and literature on approaches to distributed computing with R and current solutions in other language/systems with similar goals for statistical modelling like Python or Spark, to the sa    tisfaction of the primary supervisor.
                This was completed and can be found in the \texttt{phd} git repository under \texttt{doc/survey*}.
        \item Implement a prototype R software capable of performing operations on multiple chunks of data in parallel on different machines and use the prototype to implement one statistical model, to the satisfaction of the primary     supervisor.
                This has been satisfied through the development of the \textbf{largeScaleR} package.
        \item Attend at least 10 relevant research presentations per annum (student needs to verify participation by filling out and handing in the departmental attendance form for each presentation to the Statistics Department office    .
                This has been completed, with the talks attended summarised in table \ref{talks}, and hard copy forms with further details available at the Statistics Department Office.
        \item Participate in the Department of Statistics PhD Talks Day and/or give a departmental seminar, to the satisfaction of the main supervisor.
                Also, maintain a personal profile page (www.directory.auckland.ac.nz), providing information on scholarly activities and objectives to the satisfaction of the main supervisor and a Department of Statistics PhD Officer.
                This proposal will accompany a departmental seminar, and the personal profile page can be located at \url{https://directory.auckland.ac.nz/people/profile/jcai849}
        \item Attendance at one of the Faculty of Science Doctoral Induction Workshops.
                This was completed  on 2020-09-16.
\end{enumerate}

\begin{table}[h]
        \centering
        \begin{tabularx}{\textwidth}{llX}
                \toprule
                Date & Speaker & Title\\
                \midrule
                2020-06-19 & Malia Puloka & Posing Investigative Questions about Categorical Data - a Year 9 Case Study\\
                2020-07-22 & Yifu Tang & Likelihood Approximations for Time Series and Calibration of Approximate Bayesian Credible Sets\\
                2020-07-29 & Luke Boyle & Understanding Surgical Outcomes in New Zealand\\
                2020-11-12 & Andrew Holbrook & Bayes in the Time of Big Data\\
                2020-11-24 & Richard Perry & Modelling for COVID in Official Economic Time Series\\
                2020-11-25 & Rolf Turner & A Versatile Discrete Distribution\\
                2020-12-02 & Innocenter Amina & Integrative Analysis of High-Dimensional Data with Application to Soil Microbiome Data\\
                2021-02-25 & Charco Hui & Natural Language Processing in Clinical Trials\\
                2021-03-31 & Yehua Zang & Branching with Decision Detection\\
                Date & Speaker & Title\\
                \bottomrule
        \end{tabularx}
        \caption{\label{talks}Talks attended as part of first year goals}
\end{table}

An approximate timeline of projected future work completion is given in table \ref{future-timeline}

\begin{table}[h]
        \centering
        \begin{tabularx}{\textwidth}{lX}
                \toprule
                Date & Event\\
                \midrule
                2021-06 & Submitting a technical report for publication in XXX\\
                2021-07 & Presenting on the platform at XXX\\
                2021-12 & Platform feature development\\
                2022-02 & Publishing package on CRAN\\
                2022-05 & Development of complex statistical model demonstration for package\\
                2022-09 & Benchmarking at scale\\
                2023-05 & Thesis submission\\
                \bottomrule
        \end{tabularx}
        \caption{\label{future-timeline}A timeline of future work}
\end{table}
