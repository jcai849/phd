\documentclass[a4paper,10pt]{article}

\usepackage{doc/header}

\begin{document}
\title{DistObj Non-Assigned Data Return}
\author{Jason Cairns}
\year=2020 \month=10 \day=7
\maketitle{}

\section{Introduction}

While most operations over the distributed system explicitly produce values
that are to be assigned and saved, there comes a point, particularly with
interactive data analysis, where values are to be explicitly returned without
assignment.
Many high-level operations implicitly depend on value returns, such as
map-reduce style operations as part of the data collation prior to the reduce
step.

While the structure of assigned operations is well-defined, there remains
plenty of room for discussion of value-return architecture.

\section{Current Implementation}

As it currently stands, assignment or not is indicated through an argument to
the \mintinline{r}{do.call.distObjRef} function, with a similar procedure
being followed to assignment, in terms of messaging and the like, though
instead of returning a reference to the new distributed object,
\mintinline{r}{do.call.distObjRef} waits for a message containing values to
be returned, and returning those.
This pattern, while reasonable in the macro scale, sits on an inefficient and
ad-hoc implementation where value-containing messages pass as queues through
the redis server.
This means that an extra node is added to the journey of the value from server
to client, as well as potentially overloading the redis server and slowing all
communication down at the central point of failure.

\section{Proposed Form}

An improved form would make use of the osrv package to enable point-to-point data movement.
There is more work than just replacing messages with polling the server for the
value, due to polling being an expensive operation for both sides.
A better form would involve going through the regular assignment procedure,
then waiting for resolution and taking advantage of the existing
\mintinline{r}{emerge} function to transfer the result, followed by a new
command to delete the unnecessary values from the server.

\end{document}
