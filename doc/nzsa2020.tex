\documentclass[handout]{beamer}
\usetheme{Rochester}
\usecolortheme{seahorse}
\usepackage{doc/header}
%\usepackage{pgfpages}
%\pgfpagesuselayout{4 on 1}[a4paper,border shrink=5mm,landscape]
\title{A Platform for Large-Scale Statistical Modelling Using R}
\subtitle{Preliminary Results}
\author{Jason Cairns}
\institute{University of Auckland}
\date{NZSA Unconference 2020}
\begin{document}
\frame{\titlepage}

\section{Introduction}
	\begin{frame}
		\frametitle{Introducing a Platform for Large-Scale Analysis in R}
		\framesubtitle{Introduction \& Motivation}
		\begin{itemize}
			\item User friendly and programmer friendly distributed
				system in R for performing analyses with
				larger-than-memory data
			\item More computers \(\to\) more memory
			\item More computers \(\to\) more processing power
			\item Simply get multiple computers and run package
		\end{itemize}
	\end{frame}
	\begin{frame}
		\frametitle{A Brief Background}
		\framesubtitle{Split data into chunks}
		\begin{figure}
			\centering
			\scalebox{2}{
				\begin{tikzpicture}
				\node (rect) [draw,minimum width=10,minimum height=30] {};

				\node (midc) [object,right=of rect] {};
				\node (topc) [object,above=of midc] {};
				\node (botc) [object,below=of midc] {};

				\draw[mypath] (rect) -- (midc);
				\draw[mypath] (rect) -- (topc);
				\draw[mypath] (rect) -- (botc);
			\end{tikzpicture}
			}
		\end{figure}
	\end{frame}
	\begin{frame}
		\frametitle{A Brief Background}
		\framesubtitle{Scatter chunks across nodes}
		\begin{figure}
			\centering
			\scalebox{2}{
			\begin{tikzpicture}
				\node (midc) [object] {};
				\node (topc) [object,above=of midc] {};
				\node (botc) [object,below=of midc] {};
				
				\node (rmidc) [object,right=of midc] {};
				\node (rtopc) [object,above=of rmidc] {};
				\node (rbotc) [object,below=of rmidc] {};


				\begin{pgfonlayer}{units}
					\node[unit,fit=(rtopc)] (rtopnode) {};
					\node[unit,fit=(rmidc)] (rmidnode) {};
					\node[unit,fit=(rbotc)] (rbotnode) {};
				\end{pgfonlayer}

				\draw[mypath] (midc) -- (rmidnode);
				\draw[mypath] (topc) -- (rtopnode);
				\draw[mypath] (botc) -- (rbotnode);
			\end{tikzpicture}
			}
		\end{figure}
	\end{frame}
	\begin{frame}
		\frametitle{A Brief Background}
		\framesubtitle{Use platform to coordinate chunks/nodes through reference objects}
		\begin{figure}
			\centering
			\scalebox{2}{
			\begin{tikzpicture}
				\node (master) [object,opacity=0] {};

				\node (mc) [object,above=of master] {};
				\node (rc) [object,right=of mc] {};
				\node (lc) [object,left=of mc] {};

				\begin{pgfonlayer}{units}
					\node[unit,fit=(master)] (msn) {};
					\node[unit,fit=(mc)] (mn) {};
					\node[unit,fit=(rc)] (rn) {};
					\node[unit,fit=(lc)] (ln) {};
				\end{pgfonlayer}

				\draw[mydoublepath] (mn) -- (rn);
				\draw[mydoublepath] (mn) -- (ln);
				\draw[mydoublepath] (msn) -- (mn);
				\draw[mydoublepath] (ln.north) to[bend left] (rn.north);
				\draw[mydoublepath] (msn) -- (rn);
				\draw[mydoublepath] (msn) -- (ln);
			\end{tikzpicture}
			}
		\end{figure}
	\end{frame}
\section{Demo}
	\begin{frame}
		\frametitle{Demo}
		\framesubtitle{Simple Analysis of flights data}
		\begin{itemize}
			\item 118M observations of commercial flights in US
			\item 16Gb dataset; 32 processes on 8 low--mid-tier machines
		\end{itemize}
	\end{frame}
\section{Interface}
	\begin{frame}
		\frametitle{A User- and Programmer-friendly Interface}
		\framesubtitle{User Layer}
		\begin{itemize}
			\item For general usage
			\item Regular R (objects are just references behind the scenes)
			\item All group methods already functioning
		\end{itemize}
	\end{frame}
	\begin{frame}
		\frametitle{A User- and Programmer-friendly Interface}
		\framesubtitle{Programmer Layer}
		\begin{itemize}
			\item For extensibility and arbitrary parallel programming
			\item Asynchronous references to chunks (futures)
			\item \mintinline{r}{do.call()} derivatives;
				\mintinline{r}{do.call.distObjRef(what, args)}
			\item \mintinline{r}{emerge(ref)}
		\end{itemize}
	\end{frame}
\section{Architecture}
	\begin{frame}
		\frametitle{Architecture, by Way of Example}
		\framesubtitle{Recycling}
		\begin{figure}
			\centering
			\scalebox{2}{
			\begin{tikzpicture}
				\node (x1) [object,label=\mintinline{r}{x}] {};
				\node (x2) [object,below=0 of x1] {};
				\node (x3) [object,below=0 of x2] {};
				\node (x4) [object,below=0 of x3] {};
				\node (x5) [object,below=0 of x4] {};
				\node (x6) [object,below=0 of x5] {};

				\node (y1) [object,label=\mintinline{r}{y},right={0.5*\commdist} of x1] {};
				\node (y2) [object,below=0 of y1] {};

				\node (y3) [object,draw=dark2-3,below=0 of y2] {};
				\node (y4) [object,draw=dark2-3,below=0 of y3] {};

				\node (y5) [object,draw=dark2-1,below=0 of y4] {};
				\node (y6) [object,draw=dark2-1,below=0 of y5] {};
			\end{tikzpicture}
			}
		\end{figure}
	\end{frame}
	\begin{frame}
		\frametitle{Architecture, by Way of Example}
		\framesubtitle{A need for Communication and Movement}
		\begin{figure}
			\centering
			\scalebox{2}{
			\begin{tikzpicture}
				\node (master) [object,opacity=0] {};

				\node (x3) [object,right=\commdist of master] {};
				\node (x2) [object,above=0 of x3] {};
				\node (x1) [object,above=0 of x2] {};

				\node (x4) [object,below={3*\blockdist} of x3] {};
				\node (x5) [object,below=0 of x4] {};
				\node (x6) [object,below=0 of x5] {};

				\node (y1) [object,right={0.5*\commdist} of x1] {};
				\node (y2) [object,below=0 of y1] {};

				\begin{pgfonlayer}{units}
					\node[unit,fit=(master),label={\mintinline{r}{z <-}}] (msn) {};
					\node[unit,fit=(x1)(x2)(x3),label={\mintinline{r}{x}}] (xd1) {};
					\node[unit,fit=(x4)(x5)(x6)] (xd2) {};
					\node[unit,fit=(y1)(y2),label={\mintinline{r}{y}}] (yd1) {};
				\end{pgfonlayer}

				\path (xd1) -- (yd1) node[midway,anchor=south] {\mintinline{r}{+}};
			\end{tikzpicture}
			}
		\end{figure}
	\end{frame}
	\begin{frame}
		\frametitle{Architecture, by Way of Example}
		\framesubtitle{A means for Communication and Movement}
		\begin{figure}
			\centering
			\scalebox{2}{
			\begin{tikzpicture}
				\node (master) [object,opacity=0] {};

				\node (x3) [object,right=\commdist of master] {};
				\node (x2) [object,above=0 of x3] {};
				\node (x1) [object,above=0 of x2] {};

				\node (x4) [object,below={3*\blockdist} of x3] {};
				\node (x5) [object,below=0 of x4] {};
				\node (x6) [object,below=0 of x5] {};

				\node (y1) [object,right={0.5*\commdist} of x1] {};
				\node (y2) [object,below=0 of y1] {};

				\node (m1) [node,right=\blockdist of master,scale=0.5] {};
				\node (m2) [node,left=\blockdist of x2,scale=0.5] {};
				\node (m3) [node,left=\blockdist of x5,scale=0.5] {};

				\begin{pgfonlayer}{units}
					\node[unit,fit=(master),label={\mintinline{r}{z <-}}] (msn) {};
					\node[unit,fit=(x1)(x2)(x3),label={\mintinline{r}{x}}] (xd1) {};
					\node[unit,fit=(x4)(x5)(x6)] (xd2) {};
					\node[unit,fit=(y1)(y2),label={\mintinline{r}{y}}] (yd1) {};
				\end{pgfonlayer}

				\draw[mypath] (m1) -- (m2);
				\draw[mypath] (m1) -- (m3);
				\draw[mydoublepath] (xd1) -- (yd1);
				\draw[mydoublepath] (xd2) -- (yd1);
				\path (xd1) -- (yd1) node[midway,anchor=south] {\mintinline{r}{+}};
			\end{tikzpicture}
			}
		\end{figure}
	\end{frame}
	\begin{frame}
		\frametitle{Architecture, by Way of Example}
		\framesubtitle{Ready for the next instruction}
		\begin{figure}
			\centering
			\scalebox{2}{
			\begin{tikzpicture}
				\node (master) [object,opacity=0] {};

				\node (x3) [object,right=\commdist of master] {};
				\node (x2) [object,above=0 of x3] {};
				\node (x1) [object,above=0 of x2] {};

				\node (x4) [object,below={3*\blockdist} of x3] {};
				\node (x5) [object,below=0 of x4] {};
				\node (x6) [object,below=0 of x5] {};

				\node (z1) [object,right=\blockdist of x1] {};
				\node (z2) [object,right=\blockdist of x2] {};
				\node (z3) [object,right=\blockdist of x3] {};
				\node (z4) [object,right=\blockdist of x4] {};
				\node (z5) [object,right=\blockdist of x5] {};
				\node (z6) [object,right=\blockdist of x6] {};

				\node (y1) [object,right={0.5*\commdist} of z1] {};
				\node (y2) [object,below=0 of y1] {};

				\node (m1) [node,left=\blockdist of x2,scale=0.5] {};
				\node (m2) [node,left=\blockdist of x3,scale=0.5] {};
				\node (m3) [node,left=\blockdist of m2,scale=0.5] {};
				\node (m4) [node,left=\blockdist of x5,scale=0.5] {};
				\node (m5) [node,left=\blockdist of x6,scale=0.5] {};
				\node (m6) [node,left=\blockdist of m5,scale=0.5] {};

				\begin{pgfonlayer}{units}
					\node[unit,fit=(master)] (msn) {};
					\node[unit,fit=(x1)(x2)(x3)(z1)(z2)(z3),label={\mintinline{r}{x z}}] (xd1) {};
					\node[unit,fit=(x4)(x5)(x6)(z4)(z5)(z6)] (xd2) {};
					\node[unit,fit=(y1)(y2),label={\mintinline{r}{y}}] (yd1) {};
				\end{pgfonlayer}
			\end{tikzpicture}
			}
		\end{figure}
	\end{frame}
\section{Similar Packages}
	\begin{frame}
		\frametitle{Similar Packages have Different Focuses}
		\begin{itemize}
			\item Sparklyr - Spark + dplyr
			\item pbdDMAT - Interface to HPI platforms
			\item SNOW - Distributed processing
			\item foreach - loop interface
		\end{itemize}
	\end{frame}
\section{Summary and Questions}
	\begin{frame}
		\frametitle{Summary \& Questions}
		\begin{itemize}
			\item Highly extensible platform in R for performing
				analyses with larger-than-memory data
			\item Plenty more in store - watch this space!
			\item \url{https://github.com/jcai849/distObj.git}
			\item Questions?
		\end{itemize}
	\end{frame}
\section{Project Direction}
	\begin{frame}
		\frametitle{There is a lot More in Store}
		\begin{itemize}
			\item Resilience \& further decentralisation
			\item Memory management, GC
			\item Dashboard
		\end{itemize}
	\end{frame}
\end{document}
