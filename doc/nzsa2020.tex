\documentclass{beamer}
\usetheme{Rochester}
\usecolortheme{seahorse}
\usepackage{doc/header}
\title{A Platform for Large-Scale Statistical Modelling Using R}
\subtitle{Preliminary Results}
\author{Jason Cairns}
\institute{University of Auckland}
\date{NZSA Unconference 2020}
\begin{document}
\frame{\titlepage}

\section{Introduction}
	\begin{frame}
		\frametitle{Introduction}
		\framesubtitle{Outline \& Motivation}
		\begin{itemize}
			\item As per title
			\item User friendly and Programmer friendly platform in R for performing analyses with larger-than-memory data
			\item More computers = more memory
			\item More computers = more processing power
			\item How? 1. Get multiple computers; 2. Run package
		\end{itemize}
	\end{frame}
\section{Demo}
	\begin{frame}
		\frametitle{Demo}
		\framesubtitle{Simple Analysis of flights data}
		\begin{itemize}
			\item 118M observations of commercial flights in US
			\item 16Gb dataset
			\item How many flights from SAN? (Implicit recycling)
			\item How many flights per day of week? (Any class)
		\end{itemize}
	\end{frame}
\section{Interface}
	\begin{frame}
		\frametitle{Interface}
		\framesubtitle{Layers}
		\begin{itemize}
			\item User layer: (nearly) regular R
			\item Programmer layer: do.call derivatives, future-like references to chunks, \& emerge
			\item Lower layer: splits, combinations, custom classes \& direct chunk addressing
		\end{itemize}
	\end{frame}
\section{Architecture}
	\begin{frame}
		\frametitle{Architecture}
		\framesubtitle{Overview}
		\begin{itemize}
			\item Nodes; client-server
			\item Chunk Queues
			\item References \& Resolution for Asynchrony
		\end{itemize}
	\end{frame}
\section{Similar Packages}
	\begin{frame}
		\frametitle{Similar Packages}
		\framesubtitle{Sparklyr}
	\end{frame}
	\begin{frame}
		\frametitle{Similar Packages}
		\framesubtitle{pbdDMAT}
	\end{frame}
	\begin{frame}
		\frametitle{Similar Packages}
		\framesubtitle{SNOW \& foreach}
	\end{frame}
\section{Project Direction}
	\begin{frame}
		\frametitle{Next Steps}
		\begin{itemize}
			\item 
		\end{itemize}
	\end{frame}
	\begin{frame}
		\frametitle{Development Details}
		\begin{itemize}
		\end{itemize}
	\end{frame}
\section{Summary and Questions}
	\begin{frame}
		\frametitle{Summary \& Questions}
		\begin{itemize}
		\end{itemize}
	\end{frame}
\end{document}
