\documentclass[a4paper,10pt]{article}
\usepackage{doc/header}

\begin{document}

\begin{titlepage}
\begin{center}
	\vskip1cm
  \bfseries
  \huge UNIVERSITY OF AUCKLAND
  \vskip0.8cm
  {\LARGE Faculty of Science}
  \vskip0.5cm
  \large Department of Statistics
  \vskip1.5cm
  \Large Doctoral Research Proposal
  \vskip4cm
  \emph{\huge A Platform for Large-Scale Statistical Modelling Using R}
\end{center}

\vskip6cm

\begin{minipage}{.4\textwidth}
  \begin{flushleft}
	  \bfseries\large Supervisors:\par \emph{Dr. Simon Urbanek}\par \emph{Dr. Paul Murrell}

  \end{flushleft}
\end{minipage}
\hskip.3\textwidth
\begin{minipage}{.3\textwidth}
  \begin{flushleft}
    \bfseries\large Student:\par \emph{Jason Cairns}
  \end{flushleft}
\end{minipage}

\vskip2cm

\centering
\bfseries
\Large 2021-XX-XX
\end{titlepage}

\tableofcontents
\newpage
\section{Introduction}

Statistics is concerned with the analysis of datasets, which are continually growing bigger, and at a faster rate;
the global datasphere is expected to grow from 33 zettabytes in 2018 to 175 zettabytes by 2025\cite{rydning2018digitization}.
The scale of this growth is staggering, and continues to outpace attempts to engage meaningfully with such large datasets. 

By one measure, information storage capacity has grown at a compound annual rate of 23\% per capita over recent decades\cite{hilbert2011world}.
In spite of such massive growths in storage capacity, they are far outstripped by computational capacity over time \cite{fontana2018moore}.
Specifically, the number of components comprising an integrated circuit for computer processing have been exponentially increasing, with an additional exponential decrease in their cost\cite{moore1975progress}.
This observation, known as Moore's Law, has been the root cause for much of computational advancement over the past half-century.
The corresponding law for computer storage posits increase in bit density of storage media along with corresponding decreases in price, which has been found to track lower than expected by Moore's law metrics.
Such differentials between the generation of data, computational capacity for data processing, and constraints on data storage, have forced new techniques in computing for the analysis of large-scale data.\\

The statistician working with large datasets is constrained by the forces of increased memory and processing power, along the more overwhelming force of increased dataset size.
To take a concrete example of the problem, consider how a statistician may attempt to fit a novel model for a dataset consisting of roughly 165 million flight datapoints\cite{bot2009flights}, using methods and computational facilities typical to a small dataset.
This is actually a small dataset compared to many other large datasets, yet it is still not possible to perform an analysis in the same manner as would usually be conducted on small-scale datasets.
\texttt{R}, or any other common statistical computing system, simply won't be able to read in the data in the same fashion, as it is too big to fit in memory.
The reason for this failure lies in the memory hierarchy of computers, wherein the different forms of data storage utilised by computers have varying response times and volatility.
Using the Dell Optiplex 5080 as a typical desktop PC build, the statistician has 16Gb of Random Access Memory (RAM) for fast main memory, to be used as a program data store; and a 256Gb Solid State Drive (SSD) for slow long-term disk storage\cite{cornell2021standardcomp}.
The problem can be summed up in the need for handling datasets that are too large to fit in memory.

As a major and growing issue, there have been a plethora of responses over decades, and will be described in further detail in section \ref{background} below.  
None of the responses are entirely satisfactory for the working statistician, who may be reasonably posited to possess the following demands:

\begin{itemize}
	\item A platform that can enable the creation of novel models and apply them to larger-than-memory datasets.
	\item This platform must allow interactivity.
	\item It must be simple to use and easy to set up.
		Ideally, as close to existing systems as possible.
	\item It must be fast.
	\item It must take advantage of existing large ecosystems of statistical software.
	\item It must be robust.
	\item It must be flexible and extensible.
		A computational statistician may create custom classes and reasonably expect them to work well with the platform.
\end{itemize}

To this end, the use of the \texttt{R} programming language is a natural starting point.
The means for writing software is typically through the use of a structured, high-level programming language.
Of the myriad programming languages available, the most widespread language used for statistics is R.
In August 2020, \texttt{R} reached it's highest rank yet of 8th in the TIOBE index, a ranking of most popular programming languages, up from ranking 73rd in December 2008\cite{tiobe2021r}.
R also has a special relevance for this proposal, having been initially developed at the University of Auckland by Ross Ihaka and Robert Gentleman in 1991\cite{ihaka1996r}.

Major developments in contemporary statistical computing are typically published alongside R code implementation, usually in the form of an R package, which is a mechanism for extending R and sharing functions.
As of March 2021, the Comprehensive R Archive Network (CRAN) hosts over 17000 available packages\cite{team20:_r}.

This project seeks to build and document the statistician's large-scale modelling platform in \texttt{R}.
Preliminary results have been extremely encouraging to this endeavour, and are described in more detail in section \ref{curr} below.
There remains plenty of future work, and this is described in section \ref{future}, with tangible goals outlined in section \ref{goals}.

\section{Background}\label{background}

% Completely redo - structure by different approaches, and what packages/systems do these
\subsection{Introduction}
Statistics is concerned with the analysis of datasets, which are continually growing bigger, and at a faster rate; the global datasphere is expected to grow from 33 zettabytes in 2018 to 175 zettabytes by 2025\cite{rydning2018digitization}.

The scale of this growth is staggering, and continues to outpace attempts to engage meaningfully with such large datasets. By one measure, information storage capacity has grown at a compound annual rate of 23\% per capita over recent decades\cite{hilbert2011world}.
In spite of such massive growths in storage capacity, they are far outstripped by computational capacity over time \cite{fontana2018moore}.
Specifically, the number of components comprising an integrated circuit for computer processing have been exponentially increasing, with an additional exponential decrease in their cost\cite{moore1975progress}.
This observation, known as Moore's Law, has been the root cause for much of computational advancement over the past half-century.
The corresponding law for computer storage posits increase in bit density of storage media along with corresponding decreases in price, which has been found to track lower than expected by Moore's law metrics.
Such differentials, between the generation of data, computational capacity for data processing, and constraints on data storage, have forced new techniques in computing for the analysis of large-scale data.\\

The architecture of a computer further constrains the required approach for analysis of big data.
Most general-purpose PC's are modelled by a random-access stored-program machine, wherein a program and data are stored in registers, and data must move in and out of registers to a processing element, most commonly a Central Processing Unit (CPU). 
The movement takes at least one cycle of a computer's clock, thereby leading to larger processing time for larger data.\\

Reality dictates many different forms of data storage, with a Memory Hierarchy ranking different forms of computer storage based on their response times\cite{toy1986computer}.
The volatility of memory (whether or not it persists with no power) and the expense of faster storage forms dictates the design of commodity computers. An example of a standard build is given by the Dell Optiplex 5080, with 16Gb of Random Access Memory (RAM) for fast main memory, to be used as a program data store; and a 256Gb Solid State Drive (SSD) for slow long-term disk storage\cite{cornell2021standardcomp}.
For reasonable speed when accessing data, a program would prioritise main memory over disk storage - something not always possible when dataset size exceeds memory capacity, larger-than-memory datasets being a central issue in big data.
A program that is primarily slowed by data movement is described as I/O-bound, or memory-bound.
Much of the issue in modelling large data is the I/O-bound nature of much statistical computation.

The complement to I/O-bound computation is computation-bound, wherein the speed (or lack thereof) is determined primarily through the performance of the processing unit. This is less significant in large-scale applications than memory-bound, but remains an important design consideration when the number of computations scale with the dataset size in any nontrivial algorithm with greater than \(\mathcal{O}(1)\) complexity.

The solution to both memory- and computation-bound problems has largely been that of using more hardware; more memory, and more CPU cores. Even with this in place, more complex software is required to manage the more complex systems. As an example, with additional CPU cores, constructs such as multithreading are used to perform processing across multiple CPU cores simultaneously (in parallel).\\

The means for writing software for large-scale data is typically through the use of a structured, high-level programming language.
Of the myriad programming languages, the most widespread language used for statistics is R.
As of 2021, R increased in popularity to rank 9th in the TIOBE index.
R also has a special relevance for this proposal, having been initially developed at the University of Auckland by Ross Ihaka and Robert Gentleman in 1991\cite{ihaka1996r}.

Major developments in contemporary statistical computing are typically published alongside R code implementation, usually in the form of an R package, which is a mechanism for extending R and sharing functions.
As of March 2021, the Comprehensive R Archive Network (CRAN) hosts over 17000 available packages\cite{team20:_r}.
Several of these packages are oriented towards managing large datasets, and will be assessed in sections \ref{local} and \ref{dist}  below.
This project aims to develop an R package that provides a means for writing software to analyse very large data on clusters consisting of multiple general-purpose computers.

\subsection{Parallelism as a Strategy}
\label{parallel}
The central strategy for manipulating large datasets, from which most other patterns derive, is parallelisation. To parallelise is to engage in many computations simultaneously - this typically takes the form of either task parallelism, wherein tasks are distributed across different processors; data parallelism, where the same task operates on different pieces of data across different processors; or some mix of the two.

Parallelisation of a computational process can potentially offer speedups proportional to the number of processors available to take on work, and with recent improvements in multiprocessor hardware, the number of processors available is increasing over time.
Most general-purpose personal computers produced in the past 5 years have multiple processor cores to enable parallel computation.

Parallelism can afford major speedups, albeit with certain limitations.
Amdahl's law, formulated in 1967, aims to capture the speedup limitations, with a model derived from the argument given below in formula \ref{amdahlsform}\cite{amdahl1967law}\cite{gustafson1988law}:
\begin{equation}
	\label{amdahlsform}
	\textrm{Speedup} = \frac{1}{s+\frac{p}{N}}
\end{equation}
Where,
\begin{itemize}
	\item \(Speedup\) total speedup of whole task
	\item \(s\) time spend by serial processor on inherently serial part of program
	\item \(p\) time spent by serial processor on parallelisable part of program
	\item \(N\) number of processors
\end{itemize}
The implication is that speedup of an entire task when parallelised is granted only through the portion of the task that is otherwise constrained by singular system resources, at the proportion of execution time spent in that task.
Thus a measure of skepticism is contained in Amdahl's argument, with many tasks predicted to show no benefit to parallelise - and in reality, some likely to slow down with increased overhead given in parallelisation. 
The major response to the skepticism of Amdahl's law is given by Gustafson's law, generated from timing results in a highly parallelised system.
Gustafson's law presents a scaled speedup as per equation \ref{gustafsonsform}
\begin{equation}
	\label{gustafsonsform}
	\textrm{Scaled speedup} = s' + p'N = N + (1-N)s'
\end{equation}
Where,
\begin{itemize}
	\item \(s'\) serial time spent on the parallel system
	\item \(p'\) parallel time spent on the parallel system
\end{itemize}
This law implies far higher potential parallel speedup, varying linearly with the number of processors.

An example of an ideal task for parallelisation is the category of embarassingly parallel workload.
Such a problem is one where the separation into parallel tasks is trivial, such as performing the same operation over a dataset independently\cite{foster1995parallel}.
Many problems in statistics fall into this category, such as tabulation, monte-carlo simulation and many matrix manipulation tasks.

\subsection{Local Solutions}
\label{local}

While not specifically engaging with larger-than-memory data, a number of packages take advantage of various parallel strategies in order to process large datasets efficiently.
\textbf{multicore} is one such package, now subsumed into the \textbf{parallel} package, that grants functions that can make direct use of multiprocessor systems, thereby reducing the processing time in proportionality to the number of processors available on the system.

\textbf{data.table} also makes use of multi-processor systems, with many operations involving threading in order to rapidly perform operations on it's dataframe equivalent, the data.table.

In spite of all of these potential solutions, a major constraint remains in that only a single machine is used.
As long as there is only one machine available, bottlenecks form and no redundancy protection is offered in real-time in the event of a crash or power outage.\\

The first steps typically taken to manage larger-than-memory data is to shift part of the data into secondary storage, which generally possesses significantly more space than main memory.

This is the approach taken by the \textbf{disk.frame} package, developed by Dai ZJ.
\textbf{disk.frame} provides an eponymously named dataframe replacement class, which is able to represent a dataset far larger than RAM, constrained now only by disk size\cite{zj20}.

The mechanism of disk.frame is introduced on it's homepage with the
following explanation:

\begin{displaycquote}{zj20:_larger_ram_disk_based_data}
        {disk.frame} works by breaking large datasets into smaller
        individual chunks and storing the chunks in fst files inside a
        folder. Each chunk is a fst file containing a data.frame/data.table.
        One can construct the original large dataset by loading all the
        chunks into RAM and row-bind all the chunks into one large
        data.frame. Of course, in practice this isn't always possible; hence
        why we store them as smaller individual chunks.

                {disk.frame} makes it easy to manipulate the underlying chunks by
        implementing dplyr functions/verbs and other convenient functions
        (e.g. the (\texttt{cmap(a.disk.frame, fn, lazy = F)} function which
        applies the function fn to each chunk of a.disk.frame in parallel).
        So that {disk.frame} can be manipulated in a similar fashion to
        in-memory data.frames.
\end{displaycquote}

It works through two main principles: chunking, and an array of methods taking advantage of data.frame generics, including \textbf{dplyr} and \textbf{data.table} functions. 
Another component that isn't mentioned in the explanation, but is crucial to performance, is the parallelisation offered transparently by the package.

disk.frames are actually references to numbered \texttt{fst} files in a folder, with each file serving as a chunk. 
This is made use of through manipulation of each chunk separately, sparing RAM from dealing with a single monolithic file\cite{zj19:_inges_data}.

Fst is a means of serialising dataframes, as an alternative to RDS files\cite{klik19}. 
It makes use of an extremely fast compression algorithm developed at facebook.

Functions are usually mapped over chunks using some functional, but more complex functions such as those implementing a glm require custom solutions; as an example the direct modelling function of \texttt{dfglm()} is implemented to allow for fitting glms to the data. 
From inspection of the source code, the function is a utility wrapper for streaming disk.frame data by default into bigglm, a biglm derivative.

For grouped or aggregated functions, there is more complexity involved, due to the chunked nature of disk.frame. 
When functions are applied, they are by default applied to each chunk. 
If groups don't correspond injectively to chunks, then the syntactic chunk-wise summaries and their derivatives may not correspond to the semantic group-wise summaries expected. 
For example, summarising the median is performed by using a median-of-medians method; finding the overall median of all chunks' respective medians. 
Therefore, computing grouped medians in disk.frame result in estimates only --- this is also true of other software, such as spark, as noted in \textcite{zj19:_group_by}.

For parallelisation, future is used as the backend package, with most function mappings on chunks making use of \texttt{future::future\_lapply()} to have each chunk mapped with the intended function in parallel. 

future is initialised with access to cores through the wrapper function, \texttt{setup\_disk.frame()}\cite{zj19:_key}. 
This sets up the correct number of workers, with the minimum of workers and chunks being processed in parallel.

An important aspect to parallelisation through future is that, for purposes of cross-platform compatibility, new R processes are started for each worker\cite{zj19:_using}. 
Each process will possess it's own environment, and disk.frame makes use of future's detection capabilities to capture external variables referred to in calls, and send them to each worker.

The strategy taken by \textbf{disk.frame} has several inherent limitations, however.
\textbf{disk.frame} allows only embarassingly parallel operations for custom operations as part of a split-apply-combine (MapReduce) pattern. 
While there may theoretically be future provision for non-embarrassingly parallel operations, a significant limitation to real-time operation is the massive slowdown brought by the data movement from disk to RAM and back.

\section{Distributed Computing as a Strategy}
\label{dist}
The specs of a single contemporary commodity computer are higher than those that were used in the Apollo lunar landing, yet the management of large datasets still creates major issues, driven by a simple lack of  capacity to hold them in memory.
Supercomputers can surmount this by holding orders of magnitude higher memory, though only a few organisations or individuals can bear the financial costs of purchasing and maintaining a supercomputer. 
In a similar form, cloud computing is not a universal solution, owing to expense, security issues, and data transportation problems.
Despite this, systems rivalling supercomputers can be formed through combining many commodity computers.
An amusing illustration of this was given in 2004, when a flash mob connected hundreds of laptops to attempt running the linpack benchmark, achieving 180 gigaflops in processing output\cite{perry2004flashcomp}.

The combination of multiple independent computers to form one cohesive computing system forms part of what is known as distributed computing.
More serious efforts to connect multiple commodity computers into a larger computational system is now standard, with software such as Hadoop and Spark being commonplace in large companies for the purpose of creating distributed systems.

Distributed systems make possible the real-time manipulation of datasets larger than a single computer's RAM, by splitting up data and holding it in the RAM of multiple computers.
A factor strongly serving in favour of distributed computing is that commodity hardware exists in large quantities in most offices, oftentimes completely unused.
This means that many organisations already have the necessary base infrastructure to create a distributed system, likely only requiring some software and configuration to set it all up.
Beyond the benefit of pre-existing infrastructure, a major feature commonly offered by distributed systems, and lacking in high-powered single computer systems, is that of fault tolerance - when one computer goes down, as does happen, another computer in the system had redundant copies of much of the information of the crashed computer, and computation can resume with very little inconvenience.
A single computer, even very high-powered, doesn't usually offer fault-tolerance to this degree.

All of the packages examined the above section \ref{local} have no immediate capability to create a distributed system, and have all of the ease-of-use benefits and all of the drawbacks as discussed.

\section{Distributed Large-Scale Computing}

R does have some well-established packages used for distributed large-scale computing.
Of these, the \textbf{parallel} package is contained in the standard R image, and encapsulates \textbf{SNOW} (Simple Network Of Workstations), which provides support for distributed computing over a simple network of compputers.
The general architecture of \textbf{SNOW} makes use of a master process that holds the data and launches the cluster, pushing the data to worker processes that operate upon it and return the results to the master. \textbf{SNOW} makes use of several different communications mechanisms, including sockets or the greater MPI distributed computing library.
Some shortcomings of the described architecture is the difficulty of persisting data, meaning the expense of data transportation every time operations are requested by the master process.
In addition, as the data must originate from the master (barring generated data etc.), the master's memory size serves as a bottleneck for the whole system.\\

The \textbf{pbdR} (programming with big data in R) project provides persistent data, with the \textbf{pbdDMAT} (programming with big data Distributed MATrices) package offering a user-friendly distributed matrix class to program with over a distributed system.
It is introduced on it's main page with the
following description:
\begin{quote}
        The ``Programming with Big Data in R'' project (pbdR) is a set of highly scalable
        R packages for distributed computing and profiling in data science.

        Our packages include high performance, high-level interfaces to MPI, ZeroMQ,
        ScaLAPACK, NetCDF4, PAPI, and more. While these libraries shine brightest on
        large distributed systems, they also work rather well on small clusters and
        usually, surprisingly, even on a laptop with only two cores.

        Winner of the Oak Ridge National Laboratory 2016 Significant Event Award for
        ``Harnessing HPC Capability at OLCF with the R Language for Deep Data Science.''
        OLCF is the Oak Ridge Leadership Computing Facility, which currently includes
        Summit, the most powerful computer system in the world.\cite{pbdR2012}
\end{quote}
The project seeks especially to serve minimal wrappers around the BLAS and LAPACK
libraries along with their distributed derivatives, with the intention of
introducing as little overhead as possible.  Standard R also uses routines from
the library for most matrix operations, but suffers from numerous
inefficiencies relating to the structure of the language; for example, copies
of all objects being manipulated will be typically be created, often having
devastating performance aspects unless specific functions are used for linear
algebra operations, as discussed in \citeauthor{schmidt2017programming} (e.g.,
\texttt{crossprod(X)} instead of \texttt{t(X) \%*\% X})

Distributed linear algebra operations in pbdR depend further on the ScaLAPACK
library, which can be provided through the pbdSLAP package \cite{Chen2012pbdSLAPpackage}.
The principal interface for direct distributed computations is the pbdMPI
package, which presents a simplified API to MPI through R
\cite{Chen2012pbdMPIpackage}.  All major MPI libraries are supported, but the
project tends to make use of openMPI in explanatory documentation. A very
important consideration that isn't immediately clear  is that pbdMPI can only
be used in batch mode through MPI, rather than any interactive option as in
Rmpi \cite{yu02:_rmpi}.

The actual manipulation of distributed matrices is enabled through the pbdDMAT
package, which offers S4 classes encapsulating distributed matrices
\cite{pbdDMATpackage}. These are specialised for dense matrices through the
\texttt{ddmatrix} class, though the project offers some support for other
matrices. The \texttt{ddmatrix} class has nearly all of the standard matrix
generics implemented for it, with nearly identical syntax for all.

The package is geared heavily towards matrix operations in a statistical
programming language, so a test of it's capabilities would quite reasonably
involve statistical linear algebra. An example non-trivial routine is that of
generating data, to test randomisation capability, then fitting a generalised
linear model to the data through iteratively reweighted least squares. In this
way, not only are the basic algebraic qualities considered, but communication
over iteration on distributed objects is tested.

To work comparatively, a simple working local-only version of the algorithm is
produced in listing \ref{src:local-rwls}.

\begin{listing}
\inputminted{r}{R/review-rwls.R}
        \caption{Local GLM with RWLS}
        \label{src:local-rwls}
\end{listing}

It outputs a \(\hat{\beta}\) matrix after several seconds of computation.

Were pbdDMAT matrices to function perfectly transparently as regular matrices, , then all that would be required to convert a local algorithm to
distributed would be to prefix a \texttt{dd} to every \texttt{matrix} call, and
bracket the program with a template as per listing \ref{src:bracket}.

\begin{listing}
\begin{minted}{r}
suppressMessages(library(pbdDMAT))
init.grid()

# program code with `dd` prefixed to every `matrix` call

finalize()
\end{minted}
\caption{Idealised Common Wrap for Local to Distributed Matrices}\label{src:bracket}
\end{listing}

The program halts however, as forms of matrix creation other than through explicit \texttt{matrix()} calls are not necessarily picked up by that process; \texttt{cbind} requires a second formation of a \texttt{ddmatrix}. 
The first issue comes when performing conditional evaluation; predicates involving distributed matrices are themselves distributed matrices, and can't be mixed in logical evaluation with local predicates.

Turning local predicates to distributed matrices, then converting them all back to a local matrix for the loop to understand, finally results in a program run, however the results are still not accurate.  
This is due to \texttt{diag()<-} assignment not having been implemented, so several further changes are necessary, including specifying return type of the diag matrix as a replacement.

This serves to outline the difficulty of complete distributed transparency. 
The final working code of pbdDMAT GLM through RWLS is given in listing \ref{src:dmat}

\begin{listing}
\inputminted{r}{R/review-pbdr.R}
        \caption{pbdDMAT GLM with RWLS}
        \label{src:dmat}
\end{listing}

Decidedly more user-friendly is the \textbf{sparklyr} package, which meshes \textbf{dplyr} syntax with a \textbf{Spark} backend.
Simple analyses are made very simple (assuming a well-configured and already running \textbf{Spark} instance), but custom iterative models are extremely difficult to create through the package in spite of \textbf{Spark's} support for it. 

Given that iteration is cited by a principal author of Spark as a motivating factor in it's development when compared to Hadoop, it is reasonable to consider whether the most popular R interface to Spark, sparklyr, has support for iteration\cite{zaharia2010spark}\cite{luraschi20}.
One immediate hesitation to the suitability of sparklyr to iteration is the syntactic rooting in dplyr; dplyr is a ``Grammar of Data Manipulation'' and part of the tidyverse, which in turn is an ecosystem of packages with a shared philosophy\cite{wickham2019welcome}\cite{wickham2016r}.
The promoted paradigm is functional in nature, with iteration using for loops in R being described as ``not as important'' as in other languages; map functions from the tidyverse purrr package are instead promoted as providing greater abstraction and taking much less time to solve iteration problems.
Maps do provide a simple abstraction for function application over elements in a collection, similar to internal iterators, however they offer no control over the form of traversal, and most importantly, lack mutable state between iterations that standard loops or generators allow\cite{cousineau1998functional}.

A common functional strategy for handling a changing state is to make use of recursion, with tail-recursive functions specifically referred to as a form of iteration in \citeauthor{abelson1996structure}.
Reliance on recursion for iteration is naively non-optimal in R however, as it lacks tail-call elimination and call stack optimisations\cite{rcore2020lang}; at present the elements for efficient, idiomatic functional iteration are not present in R, given that it is not as functional a language as the tidyverse philosophy considers it to be, and sparklyr's attachment to the the ecosystem prevents a cohesive model of iteration until said elements are in place.

Iteration takes place in Spark through caching results in memory, allowing faster access speed and decreased data movement than MapReduce\cite{zaharia2010spark}.
sparklyr can use this functionality through the \texttt{tbl\_cache()} function to cache Spark dataframes in memory, as well as caching upon import with \texttt{memory=TRUE} as a formal parameter to \texttt{sdf\_copy\_to()}.
Iteration can also make use of persisting Spark Dataframes to memory, forcing evaluation then caching; performed in sparklyr through \texttt{sdf\_persist()}.

An important aspect of consideration is that sparklyr methods for dplyr generics execute through a translation of the formal parameters to Spark SQL.
This is particularly relevant in that separate Spark Data Frames can't be accessed together as in a multivariable function.
In addition, very R-specific functions such as those from the \textbf{stats} and \textbf{matrix} core libraries are not able to be evaluated, as there is no Spark SQL cognate for them.

Canned models are the only option for most users, due to \textbf{sparklyr's} reliance on Spark SQL rather than the Spark core API made available through the official \textbf{SparkR} interface.

sparklyr is excellent when used for what it is designed for.
Iteration, in the form of an iterated function, does not appear to be part of this design. 
Furthermore, all references to ``iteration'' in the primary sparklyr literature refer either to the iteration inherent in the inbuilt Spark ML functions, or the ``wrangle-visualise-model'' process popularised by Hadley Wickham\cite{luraschi2019mastering}\cite{wickham2016r}.
None of such references connect with iterated functions.

\subsection{Other Systems}

In the search for a distributed system for statistics, the world outside of R is not entirely barren.
The central issue with non-R distributed systems is that their focus is very obviously not statistics, and this shows in the level of support the platforms provide for statistical purposes.

The classical distributed system for high-performance computing is MPI.
R actually has a high-level interface to MPI through the \textbf{rmpi} package.
This package is excellent, but extremely low-level, offering little more than wrappers around MPI functions.
For the statistician who just wants to implement a model for a large dataset, such concern with minutiae is prohibitive.\\

Hadoop and Spark are two closely related systems which were mentioned earlier.

Apache Hadoop is a collection of utilities that facilitates cluster computing. 
Jobs can be sent for parallel processing on the cluster directly to the utilities using .jar files, ``streamed'' using any executable file, or accessed through language-specific APIs.

The project began in 2006, by Doug Cutting, a Yahoo employee, and Mike Cafarella. 
The inspiration for the project was a paper from Google describing the Google File System (described in \textcite{ghemawat2003google}), which was followed by another Google paper detailing the MapReduce programming model, \textcite{dean2004mapreduce}.

Hadoop consists of a file-store component, known as Hadoop Distributed File System (HDFS), and a processing component, known as MapReduce.

In operation, Hadoop splits files into blocks, then distributes them across nodes in a cluster (HDFS), where they are then processed by the node in parallel (MapReduce).
This creates the advantage of data locality, wherein data is processed by the node they exist in.

Hadoop has seen extensive industrial use as the premier big data platform upon it's release. 
In recent years it has been overshadowed by Spark, due to the greater speed gains offered by Spark for many problem sets.\\

Spark was developed with the shortcomings of Hadoop in mind;  Much of it's definition is in relation to Hadoop, which it intended to improve upon in terms of speed and usability for certain tasks\cite{zaharia2010spark}.

It's fundamental operating concept is the Resiliant Distributed Dataset (RDD), which is immutable, and generated through external data, as well as actions and transformations on prior RDD's. 
The RDD interface is exposed through an API in various languages, including R, however it appears to be abandoned to some degree, having removed from the CRAN repository at 2020-07-10 due to failing checks.

Spark requires a distributed storage system, as well as a cluster manager; both can be provided by Hadoop, among others.

Spark is known for possessing a fairly user-friendly API, intended to improve upon the MapReduce interface. 
Another major selling point for Spark is the libraries available that have pre-made functions for RDD's, including many iterative algorithms. 
The availability of broadcast variables and accumulators allow for custom iterative programming.

Spark has seen major use since it's introduction, with effectively all major big data companies having some use of Spark.\\

In the python world, the closest match to a high-level distributed system that could have statistical application is given by the python library \textbf{dask}\cite{rocklin2015dask}.
\textbf{dask} offers dynamic task scheduling through a central task graph, as well as a set of classes that encapsulate standard data manipulation structures such as NumPy arrays and Pandas dataframes. 
The main difference is that the \textbf{dask} classes take advantage of the task scheduling, including online persistence across multiple nodes.
\textbf{dask} is a large and mature library, catering to many use-cases, and exists largely in the Pythonic ``Machine Learning'' culture in comparison to the R ``Statistics'' culture.
Accordingly, the focus is more tuned to the Python software developer putting existing ML models into a large-scale capacity.
Of all the distributed systems assessed so far, \textbf{dask} comes the closest to what an ideal platform would look like for a statistician, but it misses out on the statistical ecosystem of R, provides only a few select classes, and is tied entirely to the structure of the task graph.


\section{Methodology and Approach}

The principal methodology is to utilise a research software development approach, informed by statistical ends.
This includes experimental research and rapid prototyping, along with open-source practices for public feedback.
Constraints of such an approach include the many software development practices being intended primarily for teams of software developers, which is not the case in this project, as well as the need to engage in marketing in order to have any broad feedback on an open-source project.
An immediate technical challenge that exists is the staggering array of potential technologies that could be used, coupled with the myriad niche demands of end users sitting at varying stages of their respective technologies' hype cycles.
This is typically overcome through experience but failing that, an emphasis on communication and high levels of background research can be used to manage such an uncertainty.
The central use of R also presents it's own challenges, but these are often surmounted through a foreign language interface, such as C.

Further work will continue from the existing prototype, which is described in turn in section \ref{curr} below.

\section{Preliminary Results}\label{curr}

% change master-worker, and stub terminology
% diagrams useful
\subsection{Introduction}

For meeting the problem of large scale statistical analysis in R, what is needed is a platform that is fast and robust, with a focus on a simple interface for fitting statistical models, and the flexibility for implementation of arbitrary new models within R.
As of May 2021, a prototype distributed system holding many of the described desired characteristics, has been implemented in R as part of the research.
This system is tentatively named ``LargeScaleR'', and takes the form of an R package, complete with minor documentation and a moderate proportion of tests.
It has been used to successfully read and manipulate data over a cluster of 8 nodes, including 4 processes on each node, as well as non-trivial distributed manipulations such as tabling of dataframes, all operating at a very high speed of operation.

\subsection{System Architecture}\label{sec:sys-imp}

The system operates through a modified master-worker pattern.
A master process runs as a regular R session, operated interactively by the user or by batch script.
This master process can then initialise other processes to perform work, dubbed ``worker processes''.
The worker processes are entirely independent of the master process, and none of these processes contain any information to identify other processes.
The only mechanism these processes have to communicate is via a communication queue, which serves as theprimary mechanism behind the operation of the main conceptual pieces interacted with by a user: distributed objects.

Distributed objects are a means of access to objects on a distributed system.
They serve as a reference (\textit{stub}) that acts as a transparent handle to fragmented referents (\textit{chunks}) over a distributed system.
They are effectively proxies, with generic methods passing on their standard form to the constituent chunks of the distributed object, returning another distributed object as reference to the return value of the methods acting on the chunks.
The returned distributed object is given immediately, with worker processing occuring asynchronously, giving lazy, future-like, behaviour to distributed objects.

Each chunk is a portion of data residing on some worker process.
Each has a ``descriptor'' - some unique name that exclusively references that chunk.
When they are not performing operations on a chunk, workers are monitoring all of the queues whose names correspond to the descriptors of the chunks which the respective worker holds.
Actions to be performed on the chunks are transmitted through these queues.

The master process enacts requests on these queues through methods on the distributed objects being intercepted and sent as possibly modified messages to their referent chunk queues, where they are then operated upon by the worker process.
Key to the flexibility is that the queue serves as a level of indirection, so the requesting process doesn't need to know precisely where a chunk is stored, only that it can be reached via it's queue.
This flexibility, mirroring the benefits of information hiding encouraged by message-passing object-oriented programming, allows chunks to be held arbitrarily, including on multiple nodes simultaneously.
The capacity for redundancy grants future potential for fault tolerance and resilience to nodes crashing.

A major supporting component of the system's distributed architecture is the act of ``un''-stubbing, also known as ``emerging'', wherein a reference stub is converted into it's referent.
This takes place through directly sending serialised chunks to the requester, where methods exist to combine them.

Multivariate manipulations of the data make use of unstubbing on the worker end, where multiple distributed objects are referenced in one single function request on a queue, and the worker must determine the appropriate alignment of chunks, including the use of R's recycling rules, before unstubbing all distributed objects and performing the operation.

Distributed objects stand-in for regular R objects, and can represent any class that has split and combine methods defined.
These include all atomic vectors, lists, dataframes, matrices, and arbitrary user-created classes.

Another key aspect to the architecture of the system is detailed logging, with all changes of state in a node recorded and the information dispatched to a central logger, which allows monitoring of the system in one location. The collection of logs is sufficient to build a complete picture of the system, with a Model-View-Controller pattern in an external program able to parse the logs, calculate system state, and display that in a simple interface.

\subsection{System Interface}

An exceedingly important consideration for the user is the manner in which the program is interfaced with.

As mentioned above, the LargeScaleR program is distributed as an R package, and initial setup follows standard package protocols.

% init
The system starts through getting a cluster running.
It is assumed that the hardware and network is already set up.
If the largeScaleR cluster is already running, the master can just connect directly, with some descriptive functions entered by the user allowing it to connect.
The cluster can be initialised entirely by the master session, through the use of functions taking a simple description of the intended cluster.
For ease of use, this can be given through a programmable config file describing the nature of the network, including addresses and specialised services such as a communication server and log server, as well as descriptions of the master and all the worker processes.

Upon successfully running the cluster, all processes involved will log any changes in state, including which chunks are held by which workers, and this can be viewed in an included interface.

% in
Data is initialised in the system through several different pathways.
The most straightforward for the user is to take existing data in an R session, and run a package-provided method on the data to ``stub'' it.
This serves to distribute the data as chunks across a number of worker processes, commonly referred to as ``scattering'' in MPI parlance.
This is a similar interface to that of the \textbf{SNOW} package.
While good for medium-sized data and demonstration purposes, it is unrealistic in that by definition, very large data is unable to fit in the memory of the master R session for it to be sent out.

Therefore a more standard method of initialisation when data originates from local disk is to use a package-provided reading function that streams raw data from a \texttt{csv} file or similar from disk through a root communications queue and into all workers.
This is acknowledged to be inefficient, but it currently works well under all tests when data is not already distributed.

The better method of data initialisation follows the creation of a character vector listing either url's or files local to each of the workers.
This is then distributed to the appropriate workers and an appropriate read operation is pushed to them via the distributed character vector.
This is the only method mentioned enabling full parallelisation, and is general enough to be extended for access to distributed filesystems such as HDFS.

Alongside distribution of the data, the user is returned a distributed object to use for referencing the distributed data.
In the case of the third method described, this occurs near-instantly, with the data being read concurrently.

% move
The data referenced by distributed objects can be sent from workers more simply; once established, an \texttt{unstub} method is run over a distributed object, and the underlying chunks are sent directly from the worker, to be combined at the master end.
Such movement is taken advantage of by workers as well, when they are faced with operations on multiple disparate distributed objects - this is hidden from the user, however.

% generic
The benefits of distributed objects grow commensurately with their degree of transparency, and LargeScaleR has transparency as a central goal.
Many common functions have methods provided operating on distributed objects, including most \texttt{Group} methods such as \texttt{Ops}, \texttt{Math}, and \texttt{Summary}.
More complex methods such as \texttt{table} and \texttt{rbind} are also given, and for very simple analyses, these are often enough to serve as the backbone of the analysis until the data is summarised sufficiently that it can be \texttt{unstub}b'ed and less simple analyses run locally.

% request
Alternatively, an extra layer of control is granted to the user looking for more than pre-formed functions:
functionality inspired by the \texttt{do.call} function in R allows passing anonymous or existing functions, along with a list of distributed and potentially local data, and the provided functions are run over the referent data pointed to by the distributed objects.
A distributed object referring to the results is returned.
This is actually how most of the transparent methods were implemented, with the distributed \texttt{do.call} serving as an intermediary.
Such a technique is equivalent to a Map, with a reduce also possible through either reducing at the worker end, in parallel, or even followed by an \texttt{unstub} and local reduction.

\subsection{System Implementation}\label{sec:sys-imp}

A more advanced user looking to extend or gain further understanding of the system would benefit from the knowledge of how the system is implemented.

% depends: redis, osrv, ulog, iotools
LargeScaleR is written with minimal dependencies, but takes strong advantage of the four packages it uses.
For interprocess communication, \textbf{rediscc} is used to connect to a redis server to enable queueing.
It operates extremely quickly and efficiently, with minimal overhead.
The movement of objects directly, as in \texttt{unstub} calls, takes advantage of the \textbf{osrv} package, which is specialised for rapid movement of R objects.
Logging takes advantage of \textbf{ulog}, which outputs log messages in the standard syslog format.
The \textbf{ulog} package also provides a log listening daemon.
\textbf{iotools} is used in the streaming and parsing of raw files in storage, as in the local file read mentioned above.
The dependencies have been checked and found to have no licensing issues, all possessing very free open-source licenses.

% init
The cluster is initialised using \textbf{ssh}, controlled by a programmable (sourced) configuration file at the master process.

% msg
Communication between processes uses Redis queues, with blocking pop operations to read from them.
An S3 ``message'' class was defined in LargeScaleR to standardise communication between nodes and is the only accepted form of message to be placed and parsed from a queue.

% request
Distributed objects have type ``environment'' in R, and are given S3 classes.
This is to take advantage of the mutability of environments, enabling local caching of distributed objects no matter where in the stack.
When making a request for a function to be performed on the data underlying a distributed object, the requesting process (usually master) will run a distributed \texttt{do.call} analogue, which sends the function and distributed objects wrapped in a serialised message to the queues corresponding to the descriptors of the chunks constituent to the target distributed object - if there are multiple distributed objects, currently the largest of them is chosen, but this can be generalised to any target function, including more advanced distributed scheduling.
If there are local objects included in the message, they can be sent as is through enclosing them in the \texttt{AsIs} class constructor, otherwise they are transparently stubbed with the optimisation of being sent to the same worker processes as the target distributed object.

% respond
The worker process in turn follows a simple loop of listening to queues corresponding to the chunk descriptors that it holds, as well as additional queues such as a root queue and host and process-specific queues. 
Upon popping from a queue, it evaluates the message, then possibly stores the result and goes back to listening to queues, potentially now including the queue corresponding to the descriptor of the new chunk that is stored.

The act of evaluation of a message has it's own complexities, with the following example given with two distributed objects.
% move
The distributed objects first go through a complex act of alignment.
They are all compared with the target chunk, and aligned accordingly.
If the beginning and end of the target chunk are completely outside those of the offered object, the indices of the object corresponding to those at the correct corresponding multiple are taken, and this object is then unstubbed.
In this way, recycling is implemented in a distributed fashion, with each worker determining the appropriate recycle.

Once all appropriate chunks are unstubbed, a regular \texttt{do.call} is run with the function and now-local objects

% metadata
For alignment to take place, metadata associated with each chunk, such as it's corresponding beginning and end indices, as well as it's deduced size, must be associated with a distributed object before it is made use of.
As there is purposefully no mechanism for responders to communicate with requesters, an alternative mechanism was devised; metadata requests operate just as any other distributed function call, using the standard distributed \texttt{do.call} interface, but the functions sent are unique to the chunk they are associated with, using metalinguistic evaluation to create a function that when evaluated on the worker end, sends the metadata information to a unique temporary queue, which is then listened to at the requester end, popped, and returned.


\section{Future Work}\label{future}

%\input future work

\section{Objectives \& Goals}\ref{goals}

The objective of this research project is to create a platfom for large-scale statistical computing, utilising the versatility and power of R.
Such a platform will allow statisticians to easily define and run complex distributed algorithms from within the R environment, rather than having to rely on external tools that never had statistical computation as a goal.
This platform will be demonstrated through the implementation of iterative models in R, and applying these models to real-world tasks on large-scale problems.

The following tasks will be undertaken as a part of the proposed research:

\begin{enumerate}
	\item Development of further proposed platform features, including:
		\begin{enumerate}
			\item Efficient memory usage through transient data and command chaining
			\item Fault Tolerance
			\item Interfaces with external systems
		\end{enumerate}
	\item The demonstration of such a platform through implementing complex iterative statistical models on larger-than-memory datasets
	\item Publishing the LargeScaleR package on CRAN. This requires additional development including:
		\begin{enumerate}
			\item Stability as may be expected from 90+\% test coverage
			\item Full documentation for all base functions
			\item A package vignette
			\item Passing CRAN checks
		\end{enumerate}
	\item Publishing a technical report on XXX
	\item Presenting on the platform at XXX
\end{enumerate}

\section{TODO Deliverables and Program Schedule}

%Timeline
%First year goals list, incl. table of seminars attended
% Publication, conference
% Timeline
A timeline of relevant talks attended is given in table \ref{talks}
\begin{table}
	\begin{tabular}{lll}
		\toprule
		Date & Speaker & Title\\
		\midrule
		Date & Speaker & Title\\
		\bottomrule
	\end{tabular}
	\caption{\label{talks}Talks attended as part of first year goals}
\end{table}

\section{Budget}

The development of the project itself, revolving around open-source software, does not come with any budgeting demands.
However, the field of research is rapidly moving, and requires conference attendance and presentations in order to maintain relevance.
This has been budgeted at \$1000.00 per annum, with the funds to be derived from the Postgraduate Research Student Support (PReSS) account, on an as-needed basis.
This is referenced in the Doctoral Provisional Year Review document, and is less than the budget cap of \$1200.00 for the statistics department

\printbibliography

\end{document}
