\documentclass[10pt,a4paper]{article}
\begin{document}
    \title{Evaluation in a Distributed System for R}
    \author{Jason Cairns}
    \year=2020 \month=11 \day=11
    \maketitle

% How R does evaluation; promises and laziness; call by need (memoized call by
    % name, theoreticaly beta-reduced with capture-avoiding substitution)
    % delayed evaluation, as with haskell evaluation strategy - referring to the
    % more specific parameter-passing strategy incl. evaluation order and
    % binding strategy (binding strategy = kind of value passed to function),
    % non-strict evaluation
% Other mechanisms for lazy evaluation, normal order vs. applicative order, call by value/reference
% Standard forms of evaluation
% How we've been approaching evaluation - strict, in a sense. Call by future. still evaluated in R though
% How this differs from R evaluation and what difference it makes
    % short-circuit evaluation, efficiency & scheduling. Termination.
    % e.g.	 > x = function(...) NULL
    % 		 > x(Sys.sleep(Inf))
    % call by future = parallel precompute, occasionally more efficient
    % Infinite data structures, error avoidance etc.
% Orthogonal side-point: substitution and do.call, reduction strategy, deparse
% A suggestion for emulation of R-style evaluation, difficulties (normal-order), e.g. no debuggers due to complexity
% Ease vs. equivalence vs. efficiency
% Bib:
% https://en.wikipedia.org/wiki/Lazy_evaluation
% https://en.wikipedia.org/wiki/Evaluation_strategy
% https://en.wikipedia.org/wiki/Thunk
% https://cran.r-project.org/doc/manuals/r-release/R-ints.html#Argument-evaluation
\end{document}
