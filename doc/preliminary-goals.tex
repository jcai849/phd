\documentclass[10pt,a4paper]{article}
\usepackage{biblatex}
\begin{document}

\title{Provisional Goals}
\maketitle{}

\section{Courses}
\label{sec:courses} My best options are semester 2 courses, because
the semester 1 courses coincide with the busy end of the provisional
year. I'm only considering the courses listed on the courses pages;
there are plenty more in the calendar that aren't listed as available
this year. Importantly, the goal (and the continuance of my
scholarship) is only met through a B+ grade in a course.

\subsection{Statistics}
\label{sec:statistics}

I have already taken the most relevant courses (\textbf{STATS 769}
Data Science Practice, \textbf{STATS 782} Statistical Computing, and
\textbf{STATS 760} Modern Applied Statistics)

\begin{description}
\item[STATS 730 Statistical Inference] STATS 730 gives you
  general-purpose skills that are required by many employers of
  statistical graduates. It will enable you to model real data, using
  likelihood-based statistical inference under the frequentist paradigm.
  It provides the tools and skills used by many other graduate courses
  on offer in this department, and it gives exposure to statistical
  programming in both R and the advanced optimizer TMB.\\
  \textit{This may be relevant to the ``Statistical Modelling'' part
    of the PhD project title}
\item[STATS 763 Advanced Regression Methodology] Semester one.
  Generalised linear models, generalised additive models, survival
  analysis. Smoothing and semiparametric regression. Marginal and
  conditional models for correlated data. Model selection for prediction
  and for control of confounding. Model criticism and testing.
  Computational methods for model fitting, including Bayesian
  approaches.\\
  \textit{This may also have some relevance, in a similar sense to
    STATS 730}
\item[STATS 785 Topics in Statistical Data Management] One of the key
  purposes of STATS 785 is to introduce you to the SAS software for the
  purposes of statistical inference, programming and modeling.\\
  \textit{The title was promising, but I don't see a SAS programming
    course to be relevant}
\end{description}

\subsection{Computer Science}
\label{sec:computer-science}

The courses here actually seem more directly relevant, however the
limiting factor is that I don't meet all of the prerequisites; I
graduated with a Logic and Computation degree, not Computer Science,
as I favoured the logical aspects of it. The extent of my CS courses
are a stage 3 AI course (367), and two stage 2 CS theory courses:
Algorithms (220) and Discrete Math (225). They may waive additional
prerequisites, given my experience with software development in work
and research, or they may not. A common pattern with the prerequisites
is 230 (S1/S2) \(\to\) 335 (S2) \(\to\) 7XX.

\begin{description}
\item[COMPSCI 711 Parallel and Distributed Computing] Prerequisite:
  Approval of the Academic Head or nominee. Recommended Preparation:
  COMPSCI 335. Semester 2. Computer architectures and languages for
  exploring parallelism, conceptual models of parallelism, principles
  for programming in a parallel environment, different models to achieve
  interprocess communication, concurrency control, distributed
  algorithms and fault tolerance.\\
  \textit{Seems to really go into the details of parallelism, not sure
    if this lines up with what we'll be doing. Seems very relevant,
    however}
\item[COMPSCI 732 Software Tools and Techniques] Recommended
  preparation: COMPSCI 335. Semester 1. An advanced course examining
  research issues related to tools and techniques for software design
  and development. Typical topics could include advanced software
  development methodologies, software architectures for developing
  software tools, collaborative work and software engineering, open
  source vs closed source development, issues in advanced database
  systems.\\
  \textit{I have the basics (and more) understood for all of these
    topics, with the relevance coming from the project being the
    development of software.}
\item[COMPSCI 751 Advanced Topics in Database Systems] Recommended
  preparation: COMPSCI 220, 225. Semester 1. Database principles.
  Relational model, relational algebra, relational calculus, SQL, SQL
  and programming languages, entity-relationship model, normalisation,
  query processing and query optimisation, ACID transactions,
  transaction isolation levels, database recovery, database security,
  databases and XML. Research frontiers in database systems.\\
  \textit{Databases, not exactly what I'm doing, but included because
    I actually do meet the prerequisites, and is listed as recommended
    for COMPSCI 751.}
\item[COMPSCI 752 Big Data Management] Prerequisite: Approval of the
  Academic Head or nominee. Recommended preparation: COMPSCI 351 or
  equivalent. Semester 1. Big data modelling and management in
  distributed and heterogeneous environments. Sample topics include:
  representation languages for data exchange and integration (XML and
  RDF), languages for describing the semantics of big data (DTDs, XML
  Schema, RDF Schema, OWL, description logics), query languages for big
  data (XPath, XQuery, SPARQL), data integration (Mediation via
  global-as-view and local-as-vie), large-scale search (keyword queries,
  inverted index, PageRank) and distributed computing (Hadoop,
  MapReduce, Pig), big data and blockchain technology (SPARK,
  cryptocurrency).\\
  \textit{The description has the words Hadoop, Spark, and Big Data.
    The rest seems a lot less relevant, but may be useful. I get the
    feeling that the Hadoop and Spark material will have no more than
    2 weeks dedicated to them, which won't be particularly useful to
    me.}
\end{description}

\section{Project-Related Goals}
\label{sec:proj-relat-goals}

Rosemary Barraclough advised me that while it is more common to have a
course as a goal, it is acceptable to do 2 project-related goals. An
example she gave was ``An intermediate level of experience in some
software''. This will prove to be more difficult to measure, but it
has the additional benefit of customisation to something relevant.
Also, less chance of losing my scholarship!!

Some ideas:
\begin{itemize}
\item (From Paul) complete the literature review chapter entirely by
  end of first year
\item Tests on software used, such as coursera, or cloudera
  certification
\item SICP:\@ Metaprogramming, quotation etc.?
\item \dots{}
\end{itemize}
\end{document}
