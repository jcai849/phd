\documentclass[a4paper,10pt]{article}

\usepackage{doc/header}
\begin{document}
\title{Considerations on the Value of Point-to-Point Communication}
\author{Jason Cairns}
\year=2020 \month=5 \day=22
\maketitle

\section{Introduction}

Point-to-point communication refers to communication directly from on node to
another.
It allows for fine-grained transfer of data and directives, potentially
attaining high levels of efficiency in computations that stand to gain from
such a form of communication
In addition, point-to-point communication consumes significantly less bandwidth
than that required for a complete broadcast among all nodes.

\section{Implementations}

Point-to-point communication finds its most notable implementation in MPI,
primarily through the use of \texttt{MPI\_Send} and \texttt{MPI\_Recv}
functions.
I'm not sure yet about others - Hadoop, YARN, Spark??

Applications
------------

The capacity for point-to-point communication enables many applications;
\href{https://mpitutorial.com/tutorials/point-to-point-communication-application-random-walk}{Wes
Kendall} demonstrates random walks and parallel particle tracing as some
applications that can take advantage of point-to-point communication
\cite{kendall2014mpi}.
A generalisation of such applications would be any where a large amount of
static data exists in distributed memory, and nodes perform iterative
computation on the main data in conjunction with small pieces of data that are
then transferred and received based on the output of the computation.

Applications in the domain of statistical modelling (beyond a random walk) are
unclear, but there are likely to be some in the domains of graphical models,
Bayesian networks, undirected Markov blankets, Hidden Markov Models,
potentially even neural networks wherein each node contains several consecutive
layers that are a part of a larger network spanned by all of the nodes.

\section{Conclusion}

Point-to-point communication adds huge performance capabilities to a
distributed system.
This comes at the potential for deadlock, which must be carefully managed by
both the implementation and the user.
A large-scale platform for statistical modelling in R will benefit from such a
capability, but there will be at least the following difficulties:

\begin{itemize}
	\item Determining an appropriate interface
	\item Avoiding the reinvention of the wheel
	\item Clean implementation avoiding excess communication
\end{itemize}

\printbibliography{}
\end{document}

