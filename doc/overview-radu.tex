\documentclass[a4paper,10pt]{article}

\usepackage{doc/header}

\begin{document}
\title{Project Overview}
\author{Jason Cairns}
\year=2020 \month=10 \day=20
\maketitle{}

\section{Introduction}
% motivation
How does a statistician compose and fit a novel modelling algorithm for a
dataset consisting of over 165 million flight datapoints?
More generally, how does one perform a statistical analysis over a dataset too
large to fit in computer memory?
There are many solutions to this problem, all involving a variety of tradeoffs.

% solution
What is missing is a platform that is fast and robust, with a focus on a simple
interface for fitting statistical models, and the flexibility for
implementation of arbitrary new models within R.

In this document I will provide some context of other solutions to the problem
of large datasets for R, before describing \textit{distObj}, the focus of my
project which aims to provide a powerful platform for large-scale statistical
modelling with R; the interface, architecture, and further development goals of
distObj will be explained in detail.

% demo?
\section{State of the Field}
% R packages
% keeping on disk: disk.frame
% using external system: sparklyr, pbdDMAT
% distributed computation: SNOW, foreach
% python: dask
\section{System Interface}
% future-like objects resulting from do.call methods
% make clear what do.call actually does
% distributed objects with emerge
% programmer level for manipulation of chunks
% R generic object system S3; function.class as method
\section{System Architecture}
% general layout; clients, servers, and queues
% other considerations: alignment, etc.
% journey of a task
% diagram from ...
% explanation of current system (job queues etc.)
\section{Next Steps}
% asynchrony through resolution monitoring
% robustness & resiliance
% further decentralisation
\end{document}
