\documentclass[10pt,a4paper]{article}

\usepackage{hyperref}
\usepackage{biblatex}
\addbibresource{../bib/bibliography.bib}
\usepackage{listings}
\usepackage{color} 
 
\definecolor{mygreen}{rgb}{0,0.6,0} 
\definecolor{mygray}{rgb}{0.5,0.5,0.5} 
\definecolor{mymauve}{rgb}{0.58,0,0.82} 
 
\lstset{  
  backgroundcolor=\color{white},    
  basicstyle=\ttfamily\small, 
  breaklines=true,                  
  captionpos=b,                     
  commentstyle=\color{mygreen},     
  frame=single,                     
  keepspaces=true,                  
  keywordstyle=\color{blue},        
  stringstyle=\color{mymauve},      
  tabsize=2,                        
} 


\begin{document}
\title{A Review of Iteration with sparklyr}
\author{Jason Cairns}
\year=2020 \month=5 \day=14
\maketitle{}

\section{Introduction}
Given that iteration is cited by a principal author of Spark as a motivating
factor in it's development when compared to Hadoop, it is reasonable to
consider whether the most popular R interface to Spark, sparklyr, has support
for iteration\cite{zaharia2010spark}\cite{luraschi20}.
One immediate hesitation to the suitability of sparklyr to iteration is the
syntactic rooting in dplyr; dplyr is a ``Grammar of Data Manipulation'' and part
of the tidyverse, which in turn is an ecosystem of packages with a shared
philosophy\cite{wickham2019welcome}\cite{wickham2016r}. 
The promoted paradigm is functional in nature, with iteration using for loops
in R being described as ``not as important'' as in other languages; 
map functions from the tidyverse purrr package are instead promoted as
providing greater abstraction and taking much less time to solve iteration
problems.
Maps do provide a simple abstraction for function application over elements in
a collection, similar to internal iterators, however they offer no control over
the form of traversal, and most importantly, lack mutable state between
iterations that standard loops or generators allow\cite{cousineau1998functional}.
A common functional strategy for handling a changing state is to make use of
recursion, with tail-recursive functions specifically referred to as a form of
iteration in \citeauthor{abelson1996structure}.
Reliance on recursion for iteration is naively non-optimal in R however, as it
lacks tail-call elimination and call stack optimisations\cite{rcore2020lang};
at present the elements for efficient, idiomatic functional iteration are not
present in R, given that it is not as functional a language as the tidyverse
philosophy considers it to be, and sparklyr's attachment to the the ecosystem
prevents a cohesive model of iteration until said elements are in place.

\section{Iteration}

Iteration takes place in Spark through caching results in memory, allowing
faster access speed and decreased data movement than
MapReduce\cite{zaharia2010spark}.
sparklyr can use this functionality through the \texttt{tbl\_cache()} function
to cache Spark dataframes in memory, as well as caching upon import with
\texttt{memory=TRUE} as a formal parameter to \texttt{sdf\_copy\_to()}. 
Iteration can also make use of persisting Spark Dataframes to memory, forcing
evaluation then caching; performed in sparklyr through \texttt{sdf\_persist()}.

The Babylonian method for calculating a square root is a simple iterative
procedure, used here as an example.
A standard form in R with non-optmimised initial value is given in listing
\ref{lst:basicbab}.

\begin{lstlisting}[language=R, caption={Simple Iteration with the Babylonian Method}, label=lst:basicbab]
basic_sqrt <- function(S, frac_tolerance=0.01, initial=1){
	x <- initial
	while(abs(x\^2 - S)/S > frac_tolerance){
		x <- (x + S/x)/2
	}
	x
}
\end{lstlisting}

This iterative function is trivial, but translation to sparklyr is not entirely so.

The first aspect that must be considered is that sparklyr works on Spark Data
Frames;
\texttt{x} and \texttt{S} must be copied to Spark with the aforementioned
\texttt{sdf\_copy\_to()} function.

The execution of the function in Spark is the next consideration, and sparklyr
provides two means for this to occur;
\texttt{spark\_apply()} evaluates arbitrary R code over an entire data frame.
The means of operation vary across Spark versions, ranging from launching and
running RScripts in Spark 1.5.2, to Apache Arrow conversion in Spark 3.0.0. 
The evaluation strategy of 1.5.2 is unsuitable in this instance as it is
excessive overhead to launch RScripts every iteration. 
The other form of evaluation is through using dplyr generics, which is what
will be made use of in this example.

An important aspect of consideration is that sparklyr methods for dplyr
generics execute through a translation of the formal parameters to Spark SQL.
This is particularly relevant in that separate Spark Data Frames can't be
accessed together as in a multivariable function.
In addition, very R-specific functions such as those from the \texttt{stats}
and \texttt{matrix} core libraries are not able to be evaluated, as there is no
Spark SQL cognate for them.
The SQL query generated by the methods can be accessed and ``explained''
through \texttt{show\_query()} and \texttt{explain()} respectively;
When attempting to combine two Spark Data Frames in a single query without 
joining them, \texttt{show\_query()} reveals that the Data Frame that is
referenced through the \texttt{.data} variable is translated, but the other
Data Frame has it's list representation passed through, which Spark SQL doesn't
have the capacity to parse; 
an example is given in listing \ref{lst:computer-no} (generated through listing
\ref{lst:bad}), showing an attempt to create a new column from the difference
between two seperate Data Frames

\begin{lstlisting}[language=R, caption={Attempt in R to form new column from the difference between two separate Spark data frames \texttt{S} and \texttt{x}}, label=lst:bad]
show_query(mutate(S, S = S - x)
\end{lstlisting}

\begin{lstlisting}[language=SQL, caption={Spark SQL query generated from attempt to form the difference from two seperate data frames}, label=lst:computer-no]
SELECT `S` - list(con = list(master = "yarn", method = "shell", app_name = "sparklyr", config = list(spark.env.SPARK_LOCAL_IP.local = "127.0.0.1", sparklyr.connect.csv.embedded = "\^1.*", spark.sql.legacy.utcTimestampFunc.enabled = TRUE, sparklyr.connect.cores.local = 4, spark.sql.shuffle.partitions.local = 4), state = <environment>, extensions = list(jars = character(0), packages = character(0), initializers = list(), catalog_jars = character(0)), spark_home = "/shared/spark-3.0.0-preview2-bin-hadoop3.2", backend = 4,
    monitoring = 5, gateway = 3, output_file = "/tmp/Rtmpbi2dqk/file44ec187daaf4_spark.log", sessionId = 58600, home_version = "3.0.0")) AS `S1`, `S` - list(x = "x", vars = "initial") AS `S2`
FROM `S`
\end{lstlisting}

Global variables that evaluate to SQL-friendly objects can be passed and are
evaluated prior to translation.
An example is given through listing \ref{lst:global-ok}, generated through
listing \ref{lst:ok-generator}, where the difference between a variable holding
a numeric and a Spark Data Frame is translated into the evaluation of the
variable, transformed to a float for Spark SQL, and its difference with the
Spark Data Frame, referenced directly.

\begin{lstlisting}[language=SQL, caption={Spark SQL query generated from attempt to form the difference between a data frame and a numeric}, label=lst:global-ok]
SELECT `S` - 3.0 AS `S`
FROM `S`
\end{lstlisting}

\begin{lstlisting}[language=R, caption={Capacity in sparklyr to form new column from the difference between a spark data frame and a numeric}, label=lst:ok-generator]
S
# Source: spark<S> [?? x 1]
#      S
#  <dbl>
#     9
x = 3
mutate(S, S = S - x)
# Source: spark<?> [?? x 1]
#      S
#  <dbl>
#     6
\end{lstlisting}

A reasonable approach to implementing a Babylonian method in sparklyr is then
to combine \texttt{S} and \texttt{x} in one dataframe, and iterate within
columns.

\begin{lstlisting}[language=R, caption={Babylonian method implementation using sparklyr}, label=lst:sparklyr-bab]
library(sparklyr)

sc <- spark_connect(master = "yarn")

sparklyr_sqrt <- function(S, sc, frac_tolerance=0.01, initial=1){
        bab = sdf_copy_to(sc,
                          data.frame(x=initial, S=S, unfinished=TRUE),
                          "bab", memory = TRUE, overwrite = TRUE)
	while(any(collect(bab)\$unfinished)){
                compute(mutate(bab, x = (x + S/x)/2,
                               unfinished = abs(x^2 - S)/S > frac_tolerance),
                        "bab")
        }
        collect(bab)$x
}
\end{lstlisting}

% A sneaky optimised version using global variables evaluated purely with SQL

\section{Conclusion}\label{sec:conclusion}

sparklyr is excellent when used for what it is designed for.
Iteration, in the form of an iterated function, does not appear to be part of
this design;
this was clear in the abuse required to implement a simple iterated function in
the form of the Babylonian Method.
Furthermore, all references to ``iteration'' in the primary sparklyr literature
refer either to the iteration inherent in the inbuilt Spark ML functions, or
the ``wrangle-visualise-model'' process popularised by Hadley
Wickham\cite{luraschi2019mastering}\cite{wickham2016r}.
None of such references connect with iterated functions.

Thus, it is fair to conclude that sparklyr is incapable of sensible iteration
of arbitrary R code beyond what maps directly to SQL; 
even with mutate, it is a very convoluted interface for attempting any
iteration more complex than the Babylonian Method.
Implementation of a GLM function with sparklyr iteration was initially planned,
but the point was already proven by something far simpler, and the point is one
that did not need to be laboured.

Ultimately, sparklyr is excellent at what it does, but convoluted and inefficient when
abused, as when attempting to implement iterated functions.

\printbibliography{}

\end{document}
