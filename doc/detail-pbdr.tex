\documentclass[a4paper,10pt]{article}

\usepackage{hyperref}
\usepackage{biblatex}
\addbibresource{../bib/bibliography.bib}

\begin{document}
\title{A Detail of pbdR}
\author{Jason Cairns}
\year=2020 \month=4 \day=17
\maketitle

\section{Introduction}
\label{sec:pbdr}

pbdR is a collection of packages allowing for distributed computing with
R\cite{pbdBASEpackage}. The name is an acronym for the collection's purpose;
Programming with Big Data in R. It is introduced on it's main page with the
following description:
\begin{quote}
	The ``Programming with Big Data in R'' project (pbdR) is a set of highly scalable
	R packages for distributed computing and profiling in data science.

	Our packages include high performance, high-level interfaces to MPI, ZeroMQ,
	ScaLAPACK, NetCDF4, PAPI, and more. While these libraries shine brightest on
	large distributed systems, they also work rather well on small clusters and
	usually, surprisingly, even on a laptop with only two cores.

	Winner of the Oak Ridge National Laboratory 2016 Significant Event Award for
	``Harnessing HPC Capability at OLCF with the R Language for Deep Data Science.''
	OLCF is the Oak Ridge Leadership Computing Facility, which currently includes
	Summit, the most powerful computer system in the world.\cite{pbdR2012}
\end{quote}

\section{Interface and Backend}
The project seeks to exist as a minimal wrapper around the BLAS and LAPACK
libraries, creating minimal overhead. Standard R also uses routines from the
library for most matrix operations, but suffers from numerous inefficiencies
relating to the structure of the language; for example, copies of all objects
being manipulated will be typically be created, often having devastating
performance aspects unless specific functions are used for linear algebra
operations, as discussed in \citeauthor{schmidt2017programming} (e.g.,
\texttt{crossprod(X)} instead of \texttt{t(X) \%*\% X})

Distributed linear algebra operations in pbdR depend further on the ScaLAPACK
library, provided through the pbdSLAP package \cite{Chen2012pbdSLAPpackage}.

The principal interface for distributed computations is the pbdMPI package,
which presents a simplified API to MPI through R \cite{Chen2012pbdMPIpackage}.
All major MPI libraries are supported, but in workshop materials, the project
recommends openMPI. A major note is that pbdMPI can only be used in batch mode
through MPI, rather than interactively as in Rmpi.

The actual manipulation of distributed matrices is enabled through the pbdDMAT
package, which offers S4 classes over a single program/multiple data (SPMD)
programming paradigm \cite{pbdDMATpackage}. These are specialised for dense
matrices through the \texttt{ddmatrix} class, though the project offers some
support for other matrices. The \texttt{ddmatrix} class has nearly all of the
standard matrix generics implemented for it, with nearly identical syntax for
all.

\section{Package Interaction}
The packages can be made to interact directly, for example with pbdDMAT
constructing and performing basic manipulations on distributed matrices, and
pbdMPI being used to perform further fine-tuned processing through
communicating results across nodes manually, taking advantage of the
persistence of objects at nodes through MPI.

\section{Setup}
The setup for pbdR is simple, but requires a definitely nontrivial environment
to work in, including well set up BLAS, LAPACK and derivatives, a parallel file
system with data in an appropriate format such as HDF5, and a standard MPI
library. Much of the pain of setup is ameliorated with a docker container,
provided in workshops by the project.

\printbibliography{}
\end{document}
