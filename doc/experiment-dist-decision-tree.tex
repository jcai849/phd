\documentclass[a4paper,10pt]{article}

\usepackage{hyperref}
\usepackage{biblatex}
\addbibresource{../bib/bibliography.bib}
\usepackage[svgnames]{xcolor}
\definecolor{diffstart}{named}{Grey}
\definecolor{diffincl}{named}{Green}
\definecolor{diffrem}{named}{OrangeRed}
\usepackage{listings}
\usepackage{color}

\definecolor{mygreen}{rgb}{0,0.6,0}
\definecolor{mygray}{rgb}{0.5,0.5,0.5}
\definecolor{mymauve}{rgb}{0.58,0,0.82}

\lstset{ 
	backgroundcolor=\color{white},   
		basicstyle=\ttfamily\small,
		breaklines=true,                 
		captionpos=b,                    
		commentstyle=\color{mygreen},    
		frame=single,	                   
		keepspaces=true,                 
		keywordstyle=\color{blue},       
		stringstyle=\color{mymauve},     
		tabsize=2,	                   
}
\begin{document}
\title{Experiment: Distributed Decision Tree}
\author{Jason Cairns}
\year=2020 \month=6 \day=8
\maketitle

\section{Introduction and Motivation}

The intention of this experiment is to train a decision tree on data physically
distributed across separate nodes, with the result being capable of prediction.
This is motivated by the need for the demonstration of practical use of eager
distributed objects, themselves detailed in the
\href{experiment-eager-dist-obj-pre.pdf}{Precursory} and
\href{experiment-eager-dist-obj-supp.pdf}{Supplementary} reports.

Ideally, the decision tree represents the hard part of ensemble methods such as
random forest or adaboost, and such methods could form the basis of later
experiments.

\section{Theory}

% General outline
The following brief description of decision trees follows
\citeauthor{breiman1993trees} very closely.
Decision trees operate on the principal of recursively splitting and subsetting
initial explanatory data based on predicates operating on variables within the
data, under some measure of goodness of each split; the resultant model
corresponds to a tree, with predictions on new data performed through
traversing based on the evaluation of the predicates within each node of the
tree, and the terminal nodes corrresponding to some prediction.

% questions
The predicates with which to split the data are referred to as questions, and
operate through determining whether particular variables hold specific values,
or within some other relation to a particular splitting value, such as being
less than some number for numeric data.
The tree is constructed through building the set of all possible questions,
then running the data through the questions to generate the resulting splits.
These splits are then analysed under some measure of goodness of split, then
subset by that particular predicate, and the process again applied to the
resulting subsets.

% goodness of split
The goodness of each split is drawn from the measure of lack of impurity
within the parent and two descendant nodes.
Impurity itself is a measure of differentiation among the response variables
corresponding to the subset explanatory variables; a perfectly pure dataset has
a response variable of only one type of value, and a perfectly impure dataset
has every value unique.
In this way, goodness of the split is equivalent to the minimisation of
impurity:
\begin{equation}
	\delta i (s, t) = i(t) - p_L i(t_L) - p_R i(t_R)
\end{equation}
Where \(i\) is the measure of impurity on some node \(t\), and the
conditional proportions of the data split into the left and right nodes
(\(t_L\), \(t_R\)) are given by \(p_L\) and \(p_R\) respectively.

The metric used in this case is the gini impurity measure, on the basis of ease
of computation and little variance in impurity measures as described by
Breiman.
Gini impurity is given as the probability of misclassification under arbitrary
assignment for each possible class of response variable within the subset:
\begin{equation}
	i(t) = \sum_{j \neq i} p(j \bar t) p(i \bar t)
\end{equation}
where the variables \(i\) and \(j\) stand for all possible classes belonging to
a response.
This is further simplified to:
\begin{equation}
	i(t) = 1 - \sum_j p^2(j \bar t)
\end{equation}

% pruning

\section{Implementation}

% General outline

% questions

% goodness of split

% pruning

\section{Outcome}


\printbibliography{}
\end{document}

