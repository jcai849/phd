\documentclass[10pt, a4paper]{article}
\usepackage{header}
\begin{document}
\title{Discussion on the Chunk Building Block}
\year=2021 \month=8 \day=10
\maketitle

\section{Introduction} % why use this?; datacentric etc

Over the course of prolonged prototyping, testing, and research in the construction of an \R{}-based distributed system for large-scale statistical modelling, the ``chunk'' concept has been imperative to intuitive and clean usage of the system.
This document will outline the ``chunk'', including both theory and implementation suggestions.

The term ``chunk'' is used here to refer to discrete portions of data that may be distributed across computing nodes.
It has it's concrete realisation in object-oriented terms as some instance implementing the interface of a ``chunk'' type, the description thereof expanded upon in Section \ref{int}

The provenance of the concept can be found in most distributed databases operating with distributed partitioning schemes, including MySQL Cluster and the Hadoop Distributed File System, though with a more object based approach as reflected in fragmented objects\cite{shvachko2010hadoop}\cite{kruckenberg2005mysql}.

\section{Interface}\label{int}

The chunk as a type has at a minimum two operations for it's interface: \hsrc{data()}, and \hsrc{do()}, as shown in Table \ref{tab:chunk}\tab[spec=ll,caption=Interface for the chunk type]{chunk}.
These correspond to access and execution on an abstract chunk, where \hsrc{data()} returns the underlying data encapsulated by the chunk, and \hsrc{do()} takes a function and a variable number of chunk arguments, returning a chunk representing the result.
\hsrc{do()} and \hsrc{data()} are intimately cohered, in that the \hsrc{data()} function must be called to access the underlying data for the actual calling of whatever function is given to the \hsrc{do()} function, and the result of the \hsrc{do()} operation can be accessed only through \hsrc{data()}.

\section{Suggestions for Implementation}

The implementation of such an interface strongly depends on the fact that the data underlying a chunk may be in one of several different states. 

Most notably, an instance of a \hsrc{chunk} may be returned by \hsrc{do()}, whose underlying data may still be computed either concurrently, or at some point in the future; the limitation of present data availability has purposely not been placed, in order that concurrent operation scheduling may be dynamic.
With this in mind, the chunk will adopt different behaviours internally depending on the status of the data; for example, data that is still being computed will not allow immediate access via \hsrc{data()}, and may require communications setup in order to be transferred, while fixed and pre-computed data may be immediately accessible or even cached locally.

\img{chunk1}

\section{Layering over the Chunk: An Interface for Chunk Aggregation}
\cite{guo2012parallel}
why flexibility?
matrix multiplication; search; parallel statistical

\bib{bibliography}
\end{document}
