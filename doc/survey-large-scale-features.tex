\documentclass[a4paper,10pt]{article}

\usepackage{booktabs}
\usepackage{hyperref}
\usepackage{biblatex}
\addbibresource{../bib/bibliography.bib}

\begin{document}
\title{A Survey of Large-Scale Platform Features}
\author{Jason Cairns}
\year=2020 \month=4 \day=16
\maketitle

\section{Introduction}\label{sec:intro}

To guide the development of the platform, desirable features are drawn from
existing platforms; inferred as logical extensions; and arrived at through
identification of needs. Some features are mutually exclusive, others are
suggestive of each other, but are worth considering and contrasting their
merits.

\section{Feature List}\label{sec:feature-list}

A list of features and their descriptions follows:

\begin{description}
	\item[Distributed Functionality]
		The ability to spread computation and data over separate
		computers.
		The value of distributed computing is well recognised for
		large-scale computing, in the increased capacity for
		processing, memory, and storage.
		Distributed computing typically gains latency speedup
		through parallel processing; both Amdahl's law and Gustafson's
		law give theoretical speedups for parallel jobs
		\cite{amdahl1967law} \cite{gustafson1988law}.  
		In addition, each node typically adds more working memory to the
		distributed system, allowing for larger datasets to be
		manipulated in-memory.  
		For exceedingly large datasets, the benefits of distributed
		file systems commonly allow for resiliant storage, with well-regarded
		examples including HDFS and the Google File System it is based
		upon \cite{shvachko2010hadoop} \cite{ghemawat2003google}.
	\item[Arbitrary Code Evaluation] 
		The ability to make use of user-specified code in processing.
		Most R packages for large-scale computing do enable arbitrary code,
		however they typically have some limitations, such as an inability to
		recognise global variables, as is the case with sparklyr and to a lesser
		extent future \cite{sparklyr2020limitations} \cite{microsoft20}.
		The ability for a system to adhere to a similar interface despite 
		changes in internal behaviour is useful for the sake of referential
		transparency, as well as human-computer interaction considerations
		\cite{sondergaard1990Rtda} \cite{norman2013design}.
	\item[Arbitrary Iterative Code Evaluation] The ability to process
	      user-specified code involving iteration over the whole dataset, as opposed to a
	      simple mapping.

		%% criticism of mapreduce in zaharia2010spark
	\item[Object Persistence at Nodes] The ability to retain objects at their point
	      of processing
		%% maybe see 1611.09177
	\item[Support for HDFS] Capacity to work with data and computation on the
	      Hadoop Distributed File System
		%% benefits of hdfs - shvachko2010hadoop
	\item[Ease of Setup] Is setup suitable for a computationally-focussed
	      statistician, or does it require a system administrator?
		%% R made for statisticians - rcore2020intro
	      %% lapack setup difficult (example) - quick installation guide (25 pages)
	\item[Inter-Node Communication] Can any pair of nodes communicate with each
	      other, or do they only report to a master node?
		%% benefits shown in MPI - walker1996mpi
	\item[Interactive Usage] The ability to make use of the package in an
	      interactive R session, without mandatory batch execution
		%% Especial notes on debugging
		%% repl importance from mccarthy1978history
	\item[Backend Decoupling] The implementation is maintained entirely separately
	      to the interface
		%% separation of concerns - dijkstra1982role
		%% current performant r systems - eddelbuettel2019parallel
	\item[Evaluation of Arbitrary Classes] Any class, including user-defined
	      classes can be used in evaluation
		%% limitations of just matrices as with pbdR
		%% limitations of just dataframes with anything 'tidy' and spark
		%% value of rich user-defined objects - dahl2004birth 
	\item[Direct API] The platform is explicitly programmed against at an interface
		%% contrast with generics, give examples (foreach)
	\item[Class for Generics] The platform may be set up, then programming makes
	      use of standard generics that are then utilised
	      %% contrast with API, give examples (pbdDMAT, more)
	\item[dplyr Compatible] The platform makes use of dplyr as the standard
	      generics to implicitly program over
	      %% ubiquity of dplyr (try put in something about Rstudio being the 
	      %% Red Hat of R, and dplyr being often redundant to things there 
	      %% before as well as their non-acknowledgement of everything prior 
	      %% (that typically do things better and faster), almost at a level of 
	      %% plaigiarism
\end{description}

\section{Comparison Table}\label{sec:comp-tab}

\begin{table}[h]
	\begin{tabular}{@{}llllll@{}}
		\toprule
		\multicolumn{1}{c}{Feature}         & \multicolumn{4}{c}{Platform} &                                        \\ \midrule
		                                    & RHadoop                      & Sparklyr & pbdR & disk.frame & foreach \\ \cmidrule(l){2-6}
		Distributed Functionality           &                              &          &      &            &         \\
		Arbitrary Code Evaluation           &                              &          &      &            &         \\
		Arbitrary Iterative Code Evaluation &                              &          &      &            &         \\
		Object Persistence at Nodes         &                              &          &      &            &         \\
		Support for HDFS                    &                              &          &      &            &         \\
		Ease of Setup                       &                              &          &      &            &         \\
		Inter-Node Communication            &                              &          &      &            &         \\
		Interactive Usage                   &                              &          &      &            &         \\
		Backend Decoupling                  &                              &          &      &            &         \\
		Evaluation of Arbitrary Classes     &                              &          &      &            &         \\
		Direct API                          &                              &          &      &            &         \\
		Class for Generics                  &                              &          &      &            &         \\
		dplyr Compatible                    &                              &          &      &            &         \\ \bottomrule
	\end{tabular}
	\caption{Comparison of major features among platforms\label{tab:compare-features}}
\end{table}

\printbibliography{}
\end{document}
