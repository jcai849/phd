\documentclass[a4paper,10pt]{article}

\usepackage{booktabs}
\usepackage{hyperref}
\usepackage{biblatex}
\addbibresource{../bib/bibliography.bib}

\begin{document}
\title{A Survey of Large-Scale Platform Features}
\author{Jason Cairns}
\year=2020 \month=4 \day=16
\maketitle

\section{Introduction}\label{sec:intro}

To guide the development of the platform, desirable features are drawn from
existing platforms; inferred as logical extensions; and arrived at through
identification of needs. Some features are mutually exclusive, others are
suggestive of each other, but are worth considering and contrasting their
merits.

\section{Feature List}\label{sec:feature-list}

A list of features and their descriptions follows:

\begin{description}
	\item[Distributed Functionality] The ability to spread computation and data
	      over separate computers
	\item[Arbitrary Code Evaluation] The ability to make use of user-specified code
	      in processing
	\item[Arbitrary Iterative Code Evaluation] The ability to process
	      user-specified code involving iteration over the whole dataset, as opposed to a
	      simple mapping
	\item[Object Persistence at Nodes] The ability to retain objects at their point
	      of processing
	\item[Support for HDFS] Capacity to work with data and computation on the
	      Hadoop Distributed File System
	\item[Ease of Setup] Is setup suitable for a computationally-focussed
	      statistician, or does it require a system administrator?
	\item[Inter-Node Communication] Can any pair of nodes communicate with each
	      other, or do they only report to a master node?
	\item[Backend Decoupling] The implementation is maintained entirely separately
	      to the interface
	\item[Evaluation of Arbitrary Classes] Any class, including user-defined
	      classes can be used in evaluation
	\item[Direct API] The platform is explicitly programmed against at an interface
	\item[Class for Generics] The platform may be set up, then programming makes
	      use of standard generics that are then utilised
	\item[dplyr Compatible] The platform makes use of dplyr as the standard
	      generics to implicitly program over
\end{description}

\section{Comparison Table}\label{sec:comp-tab}

\begin{table}[h]
	\begin{tabular}{@{}llllll@{}}
		\toprule
		\multicolumn{1}{c}{Feature}         & \multicolumn{4}{c}{Platform} &                                        \\ \midrule
		                                    & RHadoop                      & Sparklyr & pbdR & disk.frame & foreach \\ \cmidrule(l){2-6}
		Distributed Functionality           &                              &          &      &            &         \\
		Arbitrary Code Evaluation           &                              &          &      &            &         \\
		Arbitrary Iterative Code Evaluation &                              &          &      &            &         \\
		Object Persistence at Nodes         &                              &          &      &            &         \\
		Support for HDFS                    &                              &          &      &            &         \\
		Ease of Setup                       &                              &          &      &            &         \\
		Inter-Node Communication            &                              &          &      &            &         \\
		Backend Decoupling                  &                              &          &      &            &         \\
		Evaluation of Arbitrary Classes     &                              &          &      &            &         \\
		Direct API                          &                              &          &      &            &         \\
		Class for Generics                  &                              &          &      &            &         \\
		dplyr Compatible                    &                              &          &      &            &         \\ \bottomrule
	\end{tabular}
	\caption{Comparison of major features among platforms\label{tab:compare-features}}
\end{table}

\end{document}
