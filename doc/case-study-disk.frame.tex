\documentclass[10pt,a4paper]{article}


\usepackage{hyperref}
\usepackage{biblatex}
\addbibresource{../bib/bibliography.bib}
\usepackage{listings}
\usepackage{csquotes}

\begin{document}
\title{A Disk.Frame Case Study}
\author{Jason Cairns}
\year=2020 \month=4 \day=1
\maketitle{}

\section{Introduction}
\label{sec:introduction}

The mechanism of disk.frame is introduced on it's homepage with the
following explanation:

\begin{displaycquote}{zj20:_larger_ram_disk_based_data}
  {disk.frame} works by breaking large datasets into smaller
  individual chunks and storing the chunks in fst files inside a
  folder. Each chunk is a fst file containing a data.frame/data.table.
  One can construct the original large dataset by loading all the
  chunks into RAM and row-bind all the chunks into one large
  data.frame. Of course, in practice this isn’t always possible; hence
  why we store them as smaller individual chunks.

  {disk.frame} makes it easy to manipulate the underlying chunks by
  implementing dplyr functions/verbs and other convenient functions
  (e.g. the (\texttt{cmap(a.disk.frame, fn, lazy = F)} function which
  applies the function fn to each chunk of a.disk.frame in parallel).
  So that {disk.frame} can be manipulated in a similar fashion to
  in-memory data.frames.
\end{displaycquote}

It works through two main principles: chunking, and generic function
implementation (alongside special functions). Another component that
isn't mentioned in the explanation, but is crucial to performance, is
the parallelisation offered transparently by the package.

disk.frame is developed by Dai ZJ.

\section{Chunks and Chunking}
\label{sec:chunking}

\subsection{Chunk Representation}
\label{sec:chunk-representation}
% \item files in folder (https://diskframe.com/articles/04-ingesting-data.html#exploiting-the-structure-of-a-disk-frame)

disk.frames are actually references to numbered \texttt{fst} files in
a folder, with each file serving as a chunk. This is made use of
through manipulation of each chunk separately, sparing RAM from
dealing with a single monolithic file\cite{zj19:_inges_data}.

% \item Fst
Fst is a means of serialising dataframes, as an alternative to RDS
files\cite{klik19}. It makes use of an extremely fast compression
algorithm developed at facebook, with the R package enabling fst
written on in \href{survey-r-packages-for-local-large-scale-computing.pdf}{R
  Packages for Local Large-Scale Computing}.

% \item fst [ interface for disk.frame using fst backend, see internals
From inspection of the source code, data.table manipulations are
enabled directly through transformations of each chunk to data.tables
through the fst backend.

\subsection{Chunk Usage}
\label{sec:making-chunks}

% \item ingesting data (https://diskframe.com/articles/04-ingesting-data.html),
% \item lazy evaluation
%   (https://diskframe.com/articles/02-intro-disk-frame.html#simple-dplyr-verbs-and-lazy-evaluation)
Chunks are created transparently by disk.frame; the user can
theoretically remain ignorant of chunking. In R, the disk.frame object
serves as a reference to the chunk files. Operations on disk.frame
objects are by default lazy, waiting until the \texttt{collect()}
command to perform the collected operations and pull the chunks into R
as a single data.table. As noted in
\cite{zj19:_simpl_verbs_lazy_evaluat}, this form of lazy evaluation is
similar to the implementation of sparklyr.

Chunks are by default assigned rows through hashing the source rows,
but can be composed of individual levels of some source column, which
can provide an enormous efficiency boost for grouped operations, where
the computation visits the data, rather than the other way around.

% \item add\_chunk()
% \item get\_chunk()
% \item remove\_chunk()
Chunks can be manipulated individually, having individual ID's,
through \texttt{get\_chunk()}, as well as added or removed from
additional fst files directly, through \texttt{add\_chunk()} and
\texttt{remove\_chunk()}, respectively.

% \item rechunk()
In a computationally intensive procedure, the rows can be rearranged
between chunks based on a particular column level as a hash, through
functions such as \texttt{rechunk()}.

\section{Functions}
\label{sec:functions}

% \item constructor (as.disk.frame(), csv\_to\_disk.frame() (shard()) etc., and
%   accessors (collect())
The disk.frame object has standard procedures for construction and
access. disk.frame can be constructed from data.frames and data.tables
through \texttt{as.disk.frame()}, single or multiple csv files through
\texttt{csv\_to\_disk.frame()}, as well as zip files holding csv files.
Time can be saved later on through the application of functions to the
data during the conversion, as well as specifying what to chunk by,
keeping like data together. The process of breaking up data into
chunks is referred to by disk.frame as ``sharding'', enabled for
data.frames and data.tables through the \texttt{shard()} function.

% \item mapping: applying same function to all chunks cmap()
After creating a disk.frame object, functions can be applied directly
to all chunks invisibly through using the \texttt{cmap()} family of
functions in a form similar to base R \texttt{*apply()}

% \item dplyr verbs
A highly publicised aspect of disk.frame is the functional
cross-compatibility with dplyr verbs. These operate on disk.frame
objects lazily, and are applied through translation by disk.frame;
they are just S3 methods defined for the disk.frame class. They are
fully functioning, with the exception of \texttt{group\_by} (and it's
data.table cognate, \texttt{[by=]}, considered in more detail in
Section \ref{sec:spec-cons-group-by}.

% \item dfglm()
Beyond higher-order functions and dplyr or data.table analogues for
data manipulation, the direct modelling function of \texttt{dfglm()}
is implemented to allow for fitting glms to the data. From inspection
of the source code, the function is a utility wrapper for streaming
disk.frame data by default into bigglm, a biglm derivative.

\subsection{Grouping}
\label{sec:spec-cons-group-by}

% \item group\_by (https://diskframe.com/articles/group-by.html)
For a select set of functions, disk.frame offers a transparent grouped
\textrm{summarise()}. These are mainly composed of simple statistics
such as \texttt{mean()}, \texttt{min()}, etc.

For other grouped functions, there is more complexity involved, due to
the chunked nature of disk.frame. When functions are applied, they are
by default applied to each chunk. If groups don't correspond
injectively to chunks, then the syntactic chunk-wise summaries and
their derivatives may not correspond to the semantic group-wise
summaries expected. For example, summarising the median is performed
by using a median-of-medians method; finding the overall median of all
chunks' respective medians. Therefore, computing grouped medians in
disk.frame result in estimates only --- this is also true of other
software, such as spark, as noted in \textcite{zj19:_group_by}.

% \item chunk\_summarize(), chunk\_group\_by()
Grouped functions are thereby divided into one-stage and two-stage;
one-stage functions ``just work'' with the \texttt{group\_by()}
function, and two-stage functions requiring manual chunk aggregation
(using \texttt{chunk\_group\_by} and \texttt{chunk\_summarize}),
followed by an overall collected aggregation (using regular
\texttt{group\_by()} and \texttt{summarise()}).
\textcite{zj19:_group_by} points out that explicit two-stage approach
is similar to a MapReduce operation.

% \item custom one stage functions (https://diskframe.com/articles/custom-group-by.html )
Custom one-stage functions can be created, where user-defined chunk
aggregation and collected aggregation functions are converted into
one-stage functions by
disk.frame\cite{zj19:_custom_one_stage_group_by_funct}. These take the
forms \texttt{fn\_df.chunk\_agg.disk.frame()} and
\texttt{fn\_df.collected\_agg.disk.frame()} respectively, where
``\texttt{fn}'' is used as the name of the function, and appended to
the defined name by disk.frame, through meta-programming.

% \item hard group-by
To de-complicate the situation, but add one-off computational
overhead, chunks can be rearranged to correspond to groups, thereby
allowing for one-stage summaries just through
\texttt{chunk\_summarize()}, and exact computations of group medians.

\section{Parallelism}
\label{sec:parallelisation}

An essential component of disk.frame's speed is parallelisation; as
chunks are conceptually separate entities, function application to
each can take place with no side effects to other chunks, and can
therefore be trivially parallelised.

% \item future
For parallelisation, future is used as the backend package, with most
function mappings on chunks making use of
\texttt{future::future\_lapply()} to have each chunk mapped with the
intended function in parallel. Future is a package with complexities
in it's own right; I have written more on future in the document,
\href{detail-future.pdf}{A Detail of Future}

% \item setup\_disk.frame()
% \item https://diskframe.com/articles/concepts.html
future is initialised with access to cores through the wrapper
function, \texttt{setup\_disk.frame()}\cite{zj19:_key}. This sets up
the correct number of workers, with the minimum of workers and chunks
being processed in parallel.

% \item
%   https://diskframe.com/articles/data-table-syntax.html#external-variables-are-captured:
%   external variables are captured
An important aspect to parallelisation through future is that, for
purposes of cross-platform compatibility, new R processes are started
for each worker\cite{zj19:_using}. Each process will possess it's own
environment, and disk.frame makes use of future's detection
capabilities to capture external variables referred to in calls, and
send them to each worker.

\section{Relevance}
\label{sec:relevance}

disk.frame serves as an example of a very well-received and used
package for larger-than-RAM processing. The major keys to it's success
have been it's chart-topping performance, even in comparison with dask
and Julia, and it's user-friendliness enabled through procedural
transparency and well-articulated concepts.

disk.frame as a concept also lends itself well to extension:

The storage of chunks is currently file-based and managed by an
operating system; if fault tolerance was desired, HDFS support for
chunk storage would likely serve this purpose well.

Alternatively, OS-based file manipulation could be embraced in greater
depth, focussing on interaction with faster external tools for file
manipulation; this would lead to issues with portability, but with
reasonable checks, could enable great speedups through making use of
system utilities such as \texttt{sort} on UNIX-based systems.

The type of file may also be open to extension, with other file
formats competing for high speeds and cross-language communication
including \href{https://github.com/wesm/feather}{feather}, developed
by Wes McKinney and Hadley Wickham\cite{wes16}.

In terms of finer-grained extension, more functionality for direct
manipulation of individual chunks would potentially aid computation
when performing iterative algorithms and others of greater complexity.
% \item inspiration: transparency (find better word); disconnect between
% interface and implementation, with good access to internals
% \item good performance (https://diskframe.com/articles/vs-dask-juliadb.html)
% \item extension: HDFS storage of chunks
% \item single-chunk applications? map\_chunk\_id() map function on
%   individual chunk by id
% \item more external file operations on fst files outside R

\printbibliography{}
\end{document}
