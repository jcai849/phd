\documentclass[a4paper]{article}
\usepackage{doc/header}

\begin{document}
\title{A Review of Rmpi}
\author{Jason Cairns}
\year=2020 \month=8 \day=18
\maketitle

\section{Introduction}

Rmpi is an R package providing an interface to MPI. It is authored by XXX of XXX university and was first released in XXX.
The package serves as an extensive wrapper to MPI, providing over XXX functions, as well as R-specific provisions, such as \texttt{*apply} functions with an MPI backend.

More on MPI.

\section{Assessment of Suitability}

% Possible benefits
Given the maturity and widespread usage of MPI for High Performance Computing, utilising the platform would grant excellent capacities with an assurance of robustness.
High-speed point-to-point and mass-distributed data transfer routines are particularly ideal provisions offered by MPI, which would aid in the critical component of aligning chunks within the platform under development.
% Unsurmountable drawbacks (refer to notebook)
The drawbacks to using Rmpi will prove to be practically insurmountable however; in the case of usage on a cluster, rather than a single machine with multi-core cpu, R scripts making use of Rmpi are required to be run through an MPI program, such as openMPI's \texttt{mpirun}, rather than allowing an interactive REPL.
Even packages supporting the option to use Rmpi as a backend, such as SNOW, require the same form of operation in running an R script through an MPI program, if it is to be run on a cluster.
The same for doMPI; running on a single machine allows for interactivity, but cluster requires running through an MPI program.

\section{Future Work}

Despite the major drawback of non-interactivity, it is still worth looking into more about MPI at some point, due to it's deserved behemoth status in HPC.
The concepts and vocabulary developed by MPI has involved an enormous amount of work in terms of both manpower and time; plenty of pre-developed aspects that need not reinventing.
For such an investigation, Rmpi isn't necessarily essential; the C/C++ MPI bindings are truer to the core of MPI, starting by being the official points of development, and so the investigation is advised to start there.

\end{document}
