\documentclass[10pt, a4paper]{article}
\usepackage{header}
\begin{document}
\title{Distributed Reduce in largescaler}
\year=2021 \month=6 \day=30
\maketitle
\section{Introduction} % what is a reduce

A \hsrc{Reduce()}, commonly known as a \textit{fold}, is a computational operation evaluating a binary operation sequentially along a container of objects\cite{bird2010pearls}.
For example, in the case of a \hsrc{+} operation over a list of values drawn from \tbb, it is equivalent to a cumulative sum.
The \hsrc{Reduce()} is provided in the base \R{} distribution as part of the \textit{funprog} group of common higher-order functions.
It serves as a standard means of performing a rolling operation over some container without resorting to explicit looping, and as such is made heavy use of in the functional programming paradigm for the succinct encapsulation of the concept.
The \texttt{Reduce} referred to in the \texttt{MapReduce} paradigm is a similar, though distinct, operation, serving closer to a grouped summary\cite{dean2004mapreduce}.
The \texttt{MapReduce} is thus able to stay largely embarrassingly parallel, while a \hsrc{Reduce()} is necessarily serial.

\section{Distributed Reduce}

To create a distributed reduce using the \lsr{} system is actually mostly solved by the design of distributed objects, which can be passed to the existing \hsrc{Reduce()} function as provided in base \R{}, with no further modification.
The only additional effort is to ensure that the operant binary function is capable of operating on distributed objects.
This can be guaranteed by making use of a \hsrc{dreducable()} wrapper functional around a regular function, which returns the original function modified to operate in distributed fashion. 
The source code demonstrating this is given in Listing \ref{lst:dreduce.R}.
\src[caption={The wrapper functional providing a distributed reduce showing the very little effort required to generate a distributed reduce from the framework}]{dreduce.R}
The \hsrc{dreducable()} function is itself a simple wrapper around the \hsrc{do.ccall()} function that serves to power all of the \lsr{} distributed object operation requests.

\section{Applications and Challenges}

\bib{bibliography}
\end{document}
