\documentclass[a4paper,10pt]{article}
\usepackage{doc/header}
\begin{document}
\title{Considerations on Distributed Objects}
\author{Jason Cairns}
\year=2020 \month=6 \day=8
\maketitle

\section{Introduction}

Distributed objects are a means of access to objects on a distributed system.
They typically take the form of a reference (\textit{stub}) that acts as a transparent
handle to fragmented referents (\textit{skeletons}) over a distributed system.
Details of their methods of interaction can vary enormously; distributed
objects can exist anywhere on the spectrum of lazy/eager evaluation, for
example.
Greater transparency in distributed objects is exemplified best in R with
pbdDMAT, which provides distributed matrix objects, implementing nearly all
standard Matrix methods on them. 
pbdDMAT is discussed further in \href{review-pbdr.pdf}{Review of pbdR}.
The foundations of an implementation of distributed objects, focussing on
vectors, can be found in
\href{R/experiment-eager-dist-obj.R}{../R/experiment-eager-dist-obj.R}.

\section{Benefits}

The benefits of distributed objects grow commensurately with their degree of
transparency.
At the closest state to ideal, a distributed object would be manipulated
equivalently to its local equivalent.
More ...

\section{Contrary Recommendations}
Experience has found the state of transparency to be impossible to achieve
completely; ultimately, it is an abstraction, replete with the leaks inherent
in such a physically-dependent abstraction.

This was noted with respect to pbdR in \href{review-pbdr.pdf}{Review of pbdR}.

Further initial research has revealed strong skepticism from some
commentators\cite{waldo1996note}\cite{rotem2006fallacies},
with Martin Fowler declaring his First Law of Distributed Object Design;

\quote{``don't distribute your objects''\cite{fowler2003patterns}}

\section{Next Steps}

Despite the criticism, it is a very strong idea, with plenty of examples of
effective real world usage.  
It would be worth going into more detail on distributed objects, their
implementations in other languages (esp. CORBA), what prompted such skepticism,
and whether it is justified.

\printbibliography{}
\end{document}

