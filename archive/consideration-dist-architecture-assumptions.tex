\documentclass[10pt, a4paper]{article}

\usepackage{doc/header}

\begin{document}
\title{Distributed Architecture Informal Assumptions and Considerations}
\author{Jason Cairns}
\year=2020 \month=7 \day=26
\maketitle

\section{Introduction}
Following several weeks of development on a prototype distributed platform, a
large number of architectural choices have had to be made, covered in the
series of reports, \href{experiment-eager-dist-obj-supp.pdf}{Experiment: Eager
Distributed Object}
An attempt to outline some of the assumptions leading to these decisions is
made in section \ref{sec:assumptions}, as a means of making explicit what is
otherwise left implicit and potentially unquestioned.
Potential future assumptions with interesting consequences are considered in
section \ref{sec:potential}.

At the current iteration of this text, only a few assumptions are laid out,
however more will be progressively added following further development and
consideration.

\section{Present Assumptions}\label{sec:assumptions}

\assume{The large data made use of in the platform originates externally and
comes pre-distributed.}
If the data were instead being originated entirely locally, as in a simulation,
the memory required would soon be excessive for a single computer.
Data appropriate for this platform is too large for a single machine, so it
can't have been originated all at once from a single machine.
There may be room in the future ofor consideration of streaming data recording
and generation that distributes local \textit{ex nihilo} data, but that is a
separate concern to that of the platform for modelling on that data.
Beyond the slightly tautological argument, experience shows that most
large-scale data dealt with by a standard statistician is sourced externally.
Consequences of this assumption include the complete removal of
\texttt{as.distributed} from user-space, as it is at odds with such an
assumption.
Combined with a means of deriving locations of data chunks for import, such as
through user-provided file URI's, or hadoop locations, this enables the removal
of the concept of cluster objects from user-space.
A removal of cluster objects may lead to potential difficulties upon attempting
operations involving multiple independently-read objects, as they may be
unaligned, existing in different locations.
This leads to the consideration that alignment of distributed objects should be
an operation with side-effects, thereby ideally letting the expensive operation
of data movement occur only once between a pair of unaligned objects.
A corollary of removing cluster objects is a change in semantics; if, for
example, a library is to be loaded across all nodes on a cluster, the
declaration is no longer, ``load the library at these specific locations.''
rather it becomes, ``load the library everywhere relevent to this distributed
object.''

\assume{The platform makes use of parallelism as a means for handling large
data, in order to cope with memory constraints, and any potential speed-ups are
a secondary side-effect.}
Consider the counterfactual, that the principal concern was not large data, but
parallelism for speed: CPU-bound, not memory-bound operations.
In this kind of system, high levels of communication are acceptable and likely
beneficial.
Conversely, under our assumption, communication is required to be kept to a
minimum, given the high cost associated with transferring large swaths of data
across a network

\section{Potential Future Assumptions}\label{sec:potential}

\assume{Arbitrary classes can be distributed.}
Generalisation to arbitrary classes is an interesting pursuit, for the obvious
increases in flexibility, as well as forcing clarity in the existing concepts
of the system.
The user offering a class to the system to distribute would be required to
define methods for splitting, combining, as well as some other functions to aid
special cases such as indexing.
A marker of success would be the capacity to distribute matrices, with an
extension to different types of matrices, such as sparse, diagonal, etc.
Already some generalisation between classes is necessary;
vectors and data frames are broken into chunks for distribution using a
\texttt{split} method in order to abstract over their differences in structure.
The proposed auxiliary functions are given in table \ref{tab:aux}.

\begin{table}
	\centering
	\begin{tabularx}{\textwidth}{ l X }
		\toprule
		Function		& Purpose \\
		\midrule
		\texttt{measure} 	& A count of the number of elements 
					  within an object\\
		\texttt{split} 		& A means of breaking an object into
					  chunks\\
		\texttt{combine} 	& A means of recombining the chunks
					  locally\\
		\texttt{reftype} 	& A means of determining the 
					  appropriate class of distributed 
					  object to serve as a reference to the
					  distributed chunks \\
		\texttt{sizes} 		& A count of the number of elements 
					  within each distributed chunk\\
	\end{tabularx}
	\caption{Proposed auxiliary functions}
	\label{tab:aux}
\end{table}
			

\assume{Point-to-point communication is necessary.}
Point-to-point communication, directly between nodes without master involvement
is essential to the efficient movement of data between workers, but the
implementation involves walking a tightrope of user-friendliness.
At face value, it is antithetical to the expectation that locations should
remain unknown to the user. 
This sees a fairly simple resolution in layering the platform, with explicit
reference to locations existing only at the level of co-ordination and below;
ideally lower.
See table \ref{tab:layers} for more detail on platform layering.
An ideal outcome of point-to-point communication is the implementation of a
sorting algorithm, with a more pedestrian, but still useful outcome in direct
alignment of objects on different nodes.

\begin{table}
	\centering
	\begin{tabularx}{\textwidth}{ l X }
		\toprule
	Layer 		& Definitive Examples \\
	\midrule
	User 		& \texttt{table}, \texttt{dist.read.csv}, 
			  \texttt{dist.scan}, \texttt{Math} group (\texttt{`+`}
			  etc.), \texttt{`[<-`}\\
	Programming 	& \texttt{dist.do.call} \\
	Co-ordination 	& \texttt{which.align}, \texttt{align}, \texttt{index}\\
	Movement 	& \texttt{send}, \texttt{p2p}\\
	Communication 	& RServe\\
	\bottomrule
	\end{tabularx}
\caption{An outline of layers in the distributed architecture}
\label{tab:layers}
\end{table}

\end{document}
