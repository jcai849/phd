\documentclass[a4paper,10pt]{article}

\usepackage{doc/header}

\begin{document}
\title{Description of distObj Client-Server Call Process}
\author{Jason Cairns}
\year=2020 \month=10 \day=7
\maketitle{}

\section{Introduction}
This document aims to describe the current process enabling the evaluation of a
function over some chunks through \mintinline{r}{do.call}, as initiated by a
\mintinline{r}{do.call.distObjRef} call with a distributed object reference
in the arguments.
The process typically involves multiple nodes, with least the initial call
taking place on a node that will then act as client, and the terminal
evaluation using a node acting as a server, with nodes free to take on any
roles as appropriate.

\section{Overview}

The process is initialised on a node which will act as a client, with
\mintinline{r}{do.call.distObjRef} call, using at least one distributed
object reference in the arguments.
Of the distributed object references, one is picked as a target, for which the
nodes hosting the chunks making up the referent distributed object will serve
as the points of evaluation, with all other distributed object chunks
eventually transported to these nodes.

One message for each chunk reference within the distributed object reference is
sent to the corresponding nodes hosting the chunks.
The message contains information including the requested function, the
arguments to the function in the form of a list of distributed object
references as well as other non-distributed arguments, and the name with which
to assign the results to, which the client also keeps as an address to send
messages to for any future work on the results.
The client may continue with the remainder of its process, including producing
a future reference for the expected final results of evaluation.

Concurrent to the initialisers further work after sending a message, the node
hosting a target chunk receives the message, unpacks it and feeds the relevant
information to \mintinline{r}{do.call.msg}.

All distributed reference arguments are replaced in the list of arguments by
their actual referents.
\mintinline{r}{do.call} is then used to perform the terminal evaluation of
the given function over the argument list.
The server then assigns the value of the \mintinline{r}{do.call} to the given
chunk name within an internal chunk store environment, sending relevant
details such as size and error information back to the initial requesting node. 
The object server is also supplied with a reference to the chunk, used to send
the chunk point-to-point upon request.

\section{Argument Replacement}

The process of argument replacement on the server merits further explanation.
This procedure takes a chunk reference as the target with which to compare and
deliver the corresponding chunk of a distributed object reference.

The alignment of the argument to the target is first determined, including
recycling if necessary to fit the argument chunks appropriately to the target
chunk. The referent chunks are then emerged either from the local object store,
or if the chunks are not local, through a request to the object server of the
chunks host.
The emerged chunks are then cut and fit to the right size as per alignment
specifications, and returned, ready to replace their references in the argument
list.

\end{document}

